\documentclass{beamer}
\usetheme{Boadilla}

\makeatother
\setbeamertemplate{footline}
{
    \leavevmode%
    \hbox{%
    \begin{beamercolorbox}[wd=.4\paperwidth,ht=2.25ex,dp=1ex,center]{author in head/foot}%
        \usebeamerfont{author in head/foot}\insertshortauthor
    \end{beamercolorbox}%
    \begin{beamercolorbox}[wd=.55\paperwidth,ht=2.25ex,dp=1ex,center]{title in head/foot}%
        \usebeamerfont{title in head/foot}\insertshorttitle
    \end{beamercolorbox}%
    \begin{beamercolorbox}[wd=.05\paperwidth,ht=2.25ex,dp=1ex,center]{date in head/foot}%
        \insertframenumber{}
    \end{beamercolorbox}}%
    \vskip0pt%
}
\makeatletter
\setbeamertemplate{navigation symbols}{}

\usepackage[T1]{fontenc}
\usepackage{lmodern}
\usepackage{amssymb,amsmath}
\renewcommand{\familydefault}{\sfdefault}

\usepackage{mathtools}
\usepackage{graphicx}
\usepackage{threeparttable}
\usepackage{booktabs}
\usepackage{siunitx}
\sisetup{parse-numbers=false}

%\setlength{\OuterFrameSep}{-2pt}
%\makeatletter
%\preto{\@verbatim}{\topsep=-10pt \partopsep=-10pt }
%\makeatother

\title[Lecture 14:\ Generalized Extreme Value Models II]{Lecture 14:\ Generalized Extreme Value Models II}
\author[ResEcon 703:\ Advanced Econometrics]{ResEcon 703:\ Topics in Advanced Econometrics}
\date{Matt Woerman\\University of Massachusetts Amherst}

\begin{document}

{\setbeamertemplate{footline}{} 
\begin{frame}[noframenumbering]
    \titlepage
\end{frame}
}

\begin{frame}\frametitle{Agenda}
    Last time
    \begin{itemize}
        \item Generalized Extreme Value Models
    \end{itemize}
    \vspace{2ex}
    Today
    \begin{itemize}
        \item Generalized Extreme Value Models Example in R
    \end{itemize}
    \vspace{2ex}
    Upcoming
    \begin{itemize}
        \item Reading for next time
        \begin{itemize}
            \item Train textbook, Chapter 6
        \end{itemize}
        \item Problem sets
        \begin{itemize}
            \item Problem Set 3 is posted, due October 31
        \end{itemize}
    \end{itemize}
\end{frame}

\begin{frame}\frametitle{Nested Logit}
    The nested logit model relaxes the (sometimes overly) strong assumption of the logit model
    \begin{itemize}
    	\item Nested logit allows for the unobserved components of utility ($\varepsilon_{nj}$) to be correlated between alternatives for the same decision maker
    \end{itemize}
    \vspace{2ex}
    The nested logit model groups alternatives into ``nests''
    \begin{itemize}
    	\item $Cov(\varepsilon_{nj}, \varepsilon_{nm}) = 0$ if $j$ and $m$ are in different nests
    	\item $Cov(\varepsilon_{nj}, \varepsilon_{nm}) \geq 0$ if $j$ and $m$ are in the same nest
    \end{itemize}
    \vspace{2ex}
    These correlations relax the substitution patterns of the logit model
    \begin{itemize}
    	\item IIA holds for two alternatives within the same nest
    	\item IIA does not necessarily hold for alternatives in different nests
    \end{itemize}
\end{frame}

\begin{frame}\frametitle{}
    \vfill
    \centering
    \begin{beamercolorbox}[center]{title}
        \Large Generalized Extreme Value Model Example in R
    \end{beamercolorbox}
    \vfill
\end{frame}

\begin{frame}\frametitle{Nested Logit Model Example}
    We are again studying how consumers make choices about expensive and highly energy-consuming systems in their homes. We have data on 250 households in California and the type of HVAC (heating, ventilation, and air conditioning) system in their home. Each household has the following choice set, and we observe the following data \\
    \vspace{3ex}
    \begin{columns}
    	\begin{column}{0.5\textwidth}
		    Choice set
		    \begin{itemize}
		    	\item GCC: gas central with AC
		    	\item ECC: electric central with AC
		    	\item ERC: electric room with AC
		    	\item HPC: heat pump with AC
		    	\item GC: gas central
		    	\item EC: electric central
		    	\item ER: electric room
		    \end{itemize}
		    \vspace{2ex}
	    \end{column}
	    \begin{column}{0.5\textwidth}
		    Alternative-specific data
		    \begin{itemize}
		    	\item ICH: installation cost for heat
		    	\item ICCA: installation cost for AC
		    	\item OCH: operating cost for heat
		    	\item OCCA: operating cost for AC
		    \end{itemize}
		    \vspace{2ex}
		    Household demographic data
		    \begin{itemize}
		    	\item income: annual income
		    \end{itemize}
		\end{column}
    \end{columns}
\end{frame}

\begin{frame}[fragile]\frametitle{Load Dataset}
    <<R CODE HERE>>
\end{frame}

\begin{frame}[fragile]\frametitle{Dataset}
    <<R CODE HERE>>
\end{frame}

\begin{frame}\frametitle{Nested Logit Models to Estimate with \texttt{mlogit()}}
    The representative utility of each alternative is
    $$V_{nj} = \alpha_j + \beta_1 IC_{nj} + \beta_2 OC_{nj}$$ \\
    \vspace{3ex}
    Nesting structures to consider
    \begin{itemize}
    	\item Air conditioning vs.\ heating only with unequal within-nest correlation
    	\item Air conditioning vs.\ heating only with equal within-nest correlation
    	\item Electric vs.\ gas with unequal within-nest correlation
    	\item Pairwise combinatorial logit 
    \end{itemize}
\end{frame}

\begin{frame}[fragile]\frametitle{Nested Logit Models in R}
    <<R CODE HERE>>
    \vspace{2ex}
    \texttt{mlogit()} arguments for nested logit
    \begin{enumerate}
        \item \texttt{formula, data, reflevel}: same as a multinomial logit model
        \item \texttt{nests}: named list of character vectors that defines nests, or \texttt{`pcl'}
        \item \texttt{un.nest.el}: \texttt{TRUE} forces all nests to have equal correlation, \texttt{FALSE} estimates a different correlation for each nest
        \item \texttt{heterosc}: \texttt{TRUE} estimates a heteroskedastic logit model
    \end{enumerate}
\end{frame}

\begin{frame}[fragile]\frametitle{Format Dataset in a Long Format}
    <<R CODE HERE>>
\end{frame}

\begin{frame}[fragile]\frametitle{Dataset in a Long Format}
    <<R CODE HERE>>
\end{frame}

\begin{frame}[fragile]\frametitle{Clean Dataset}
    <<R CODE HERE>>
\end{frame}

\begin{frame}[fragile]\frametitle{Cleaned Dataset}
    <<R CODE HERE>>
\end{frame}

\begin{frame}[fragile]\frametitle{Convert Dataset to \texttt{mlogit} Format}
    <<R CODE HERE>>
\end{frame}

\begin{frame}[fragile]\frametitle{Dataset in \texttt{mlogit} Format}
    <<R CODE HERE>>
\end{frame}

\begin{frame}[fragile]\frametitle{Nests for Air Conditioning Vs.\ Heating Only}
    <<R CODE HERE>>
\end{frame}

\begin{frame}[fragile]\frametitle{Model Results with AC and Heat Nests}
    \vspace{1ex}
    <<R CODE HERE>>
\end{frame}

\begin{frame}[fragile]\frametitle{Nests for AC Vs.\ Heat Only with Common Correlation}
    <<R CODE HERE>>
\end{frame}

\begin{frame}[fragile]\frametitle{Model Results with Common Correlation}
    \vspace{1ex}
    <<R CODE HERE>>
\end{frame}

\begin{frame}[fragile]\frametitle{Likelihood Ratio Test of Correlation Parameters}
    Test that the correlation parameters are equal
    $$H_0 \text{: } \lambda_k = \lambda_{\ell}$$ \\
    \vspace{2ex}
    <<R CODE HERE>>
\end{frame}

\begin{frame}[fragile]\frametitle{Nests for Electric Vs.\ Gas}
    <<R CODE HERE>>
\end{frame}

\begin{frame}[fragile]\frametitle{Model Results with Electric and Gas Nests}
    \vspace{1ex}
    <<R CODE HERE>>
\end{frame}

\begin{frame}\frametitle{Likelihood Ratio Test of Nesting Structure}
    Can we test if the cooling vs.\ heating-only nests are better than the electric vs.\ gas nests?
    \begin{itemize}
        \item Likelihood ratio test compares a restricted model to its unrestricted counterpart
        \item Is there any way to write down one model as a restriction of the other?
        \item If not, then we cannot use a likelihood ratio test in this case
    \end{itemize}
\end{frame}

\begin{frame}[fragile]\frametitle{Paired Combinatorial Logit}
    <<R CODE HERE>>
\end{frame}

\begin{frame}[fragile]\frametitle{Model Results for Paired Combinatorial Logit}
    \vspace{1ex}
    <<R CODE HERE>>
\end{frame}

\begin{frame}[fragile]\frametitle{Paired Combinatorial Logit with Normalization}
    <<R CODE HERE>>
\end{frame}

\begin{frame}[fragile]\frametitle{Model Results for Normalized Paired Combinatorial Logit}
    \vspace{1ex}
    <<R CODE HERE>>
\end{frame}

\begin{frame}\frametitle{Announcements}
    Reading for next time
    \begin{itemize}
        \item Train textbook, Chapter 6
    \end{itemize}
    \vspace{3ex}
    Upcoming
    \begin{itemize}
        \item Problem Set 3 is posted, due October 31
    \end{itemize}
\end{frame}

\end{document}