\documentclass{beamer}
\usetheme{Boadilla}

\makeatother
\setbeamertemplate{footline}
{
    \leavevmode%
    \hbox{%
    \begin{beamercolorbox}[wd=.4\paperwidth,ht=2.25ex,dp=1ex,center]{author in head/foot}%
        \usebeamerfont{author in head/foot}\insertshortauthor
    \end{beamercolorbox}%
    \begin{beamercolorbox}[wd=.55\paperwidth,ht=2.25ex,dp=1ex,center]{title in head/foot}%
        \usebeamerfont{title in head/foot}\insertshorttitle
    \end{beamercolorbox}%
    \begin{beamercolorbox}[wd=.05\paperwidth,ht=2.25ex,dp=1ex,center]{date in head/foot}%
        \insertframenumber{}
    \end{beamercolorbox}}%
    \vskip0pt%
}
\makeatletter
\setbeamertemplate{navigation symbols}{}

\usepackage[T1]{fontenc}
\usepackage{lmodern}
\usepackage{amssymb,amsmath}
\renewcommand{\familydefault}{\sfdefault}

\DeclareMathOperator*{\argmin}{argmin}
\DeclareMathOperator*{\plim}{plim}

\usepackage{mathtools}
\usepackage{graphicx}
\usepackage{threeparttable}
\usepackage{booktabs}
\usepackage{siunitx}
\sisetup{parse-numbers=false}

%\setlength{\OuterFrameSep}{-2pt}
%\makeatletter
%\preto{\@verbatim}{\topsep=-10pt \partopsep=-10pt }
%\makeatother

\title[Lecture 11:\ Generalized Method of Moments I]{Lecture 11:\ Generalized Method of Moments I}
\author[ResEcon 703:\ Advanced Econometrics]{ResEcon 703:\ Topics in Advanced Econometrics}
\date{Matt Woerman\\University of Massachusetts Amherst}

\begin{document}

{\setbeamertemplate{footline}{} 
\begin{frame}[noframenumbering]
    \titlepage
\end{frame}
}

\begin{frame}\frametitle{Agenda}
    Last time
    \begin{itemize}
        \item Nonlinear Least Squares Example in R
    \end{itemize}
    \vspace{2ex}
    Today
    \begin{itemize}
    	\item Method of Moments
    	\item Generalized Method of Moments
    	\item Properties of the Generalized Method of Moments Estimator
    	\item Optimal Generalized Method of Moments
    \end{itemize}
    \vspace{2ex}
    Upcoming
    \begin{itemize}
    	\item No class on Tuesday! Monday schedule.
        \item Reading for next time
        \begin{itemize}
            \item Optional: Crawford and Yurukoglu (2012)
        \end{itemize}
        \item Problem sets
        \begin{itemize}
            \item Problem Set 2 is posted, due October 17
        \end{itemize}
    \end{itemize}
\end{frame}

\begin{frame}\frametitle{MLE and NLS Recap}
    Maximum likelihood estimation
    \begin{itemize}
    	\item Requires a strong assumption about the density of our data
    	\item Has many nice properties (lowest-variance consistent estimator)
    \end{itemize}
    \vspace{3ex}
    Nonlinear least squares
    \begin{itemize}
    	\item Relies on weaker distributional assumptions
    	\item Requires a single ``regression equation'' to be estimated
    \end{itemize}
    \vspace{3ex}
    But what if we do not want to make a strong assumption about our data, and we want to jointly estimate multiple nonlinear equations at once?
\end{frame}

\begin{frame}\frametitle{}
    \vfill
    \centering
    \begin{beamercolorbox}[center]{title}
        \Large Method of Moments
    \end{beamercolorbox}
    \vfill
\end{frame}

\begin{frame}\frametitle{(Generalized) Method of Moments}
    Method of moments and generalized method of moments
    \begin{itemize}
    	\item Common in modern empirical economics
    	\begin{itemize}
    		\item Most common in industrial organization, macroeconomics, and finance
    	\end{itemize}
    	\item More flexible than MLE or NLS
    	\begin{itemize}
    		\item A few weak assumptions required! (Other than moment conditions\ldots)
    	\end{itemize}
    	\item (Almost?) All estimators you have learned so far are GMM estimators
    	\begin{itemize}
    		\item OLS, 2SLS, GLS, MLE, NLS, etc.
    	\end{itemize}
    \end{itemize}
    \vspace{2ex}
    Overview of MM and GMM
    \begin{itemize}
    	\item MM and GMM use population moment conditions to estimate parameters
    	\begin{itemize}
    		\item Moments can come from economic models, model fitting criteria, etc.
    	\end{itemize}
    	\item MM and GMM estimators ``solve'' the sample moment conditions that are analogous to these population moment conditions
    \end{itemize}
\end{frame}

\begin{frame}\frametitle{Moment Conditions}
    Moment conditions are functions of parameters and data that equal zero in expectation when evaluated at the true parameter values
    \begin{alignat*}{3}
    	\text{Mean:} \; &\mu_y = E[y] \; &&\Rightarrow \; &&E[y - \mu_y] = 0 \\
    	\text{Var:} \; &\sigma_y^2 = E[(y - \mu)^2] \; &&\Rightarrow \; &&E[ (y - \mu)^2 - \sigma_y^2] = 0 \\
    	\text{Cov:} \; &\sigma_{xy} = E[(y - \mu_y) (x - \mu_x)] \; &&\Rightarrow \; &&E[ (y - \mu_y) (x - \mu_x) - \sigma_{xy}] = 0
    \end{alignat*} \\
    \vspace{2ex}
    How do we normally generate moment conditions?
    \begin{itemize}
    	\item Economic models
    	\begin{itemize}
    		\item First-order conditions always set something equal to zero
    	\end{itemize}
    	\item Econometric or statistical model
    	\begin{itemize}
    		\item Example: Instruments must be uncorrelated with errors
    	\end{itemize}
    	\item Model fit
    	\begin{itemize}
    		\item Example: Predicted market shares must equal realized market shares
    	\end{itemize}
    \end{itemize}
\end{frame}

\begin{frame}\frametitle{Method of Moments Estimator}
    We have $K$ population moment conditions for $K$ parameters
    \begin{align*}
    	E[m(w_i, \theta)] &= 0 \\
    	\intertext{where $m(\cdot)$ is a vector of $K$ functions, $\theta$ is a vector of $K$ parameters, and $w_i$ is observable data (outcome, explanatory variables, instruments, etc.)}
    	\intertext{The method of moments estimator uses the sample analogs of these moment conditions }
    	\frac{1}{n} \sum_{i = 1}^n m(w_i, \theta) & \\
    	\intertext{The method of moments estimator is the set of parameters that solves}
    	\frac{1}{n} \sum_{i = 1}^n m(w_i, \hat{\theta}) &= 0
    \end{align*}
\end{frame}

\begin{frame}\frametitle{MM Example: Population Mean}
    Under the assumption that observations are i.i.d, sample means can be used to estimate population means \\
    \vspace{2ex}
	If $y$ is distributed i.i.d.\ with mean $\mu$, then a population moment condition is
	\begin{align*}
		E[y - \mu] &= 0 \\
		\intertext{Replacing the expectation with the sample analog gives}
		\frac{1}{n} \sum_{i = 1}^n (y_i - \hat{\mu}) &= 0
		\intertext{Solving this one equation for the one parameter yields}
		\hat{\mu} = \frac{1}{n} \sum_{i = 1}^n y_i &= \bar{y}
	\end{align*}
\end{frame}

\begin{frame}\frametitle{MM Example: Linear Regression}
    Ordinary least squares is a special case of a MM estimator \\
    \vspace{2ex}
    Consider the most general linear regression model
    $$y_i = \beta' x_i + \varepsilon_i$$
    where $\beta$ is a vector of $K$ parameters and $E[\varepsilon_i \mid x_i] = 0$. Then $E[x_i \varepsilon_i] = 0$ (from last semester), which gives $K$ population moment conditions
    \begin{align*}
    	E[x_i (y_i - \beta' x_i)] &= 0 \\
    	\intertext{Replacing the expectation with the sample analog gives}
    	\frac{1}{n} \sum_{i = 1}^n x_i (y_i - \hat{\beta}' x_i) &= 0
    \end{align*}
    Solving these $K$ equations for the $K$ parameters yields
    $$\hat{\beta} = \left(\sum_{i = 1}^n x_i x_i' \right)^{-1} \left(\sum_{i = 1}^n x_i y_i \right)$$
\end{frame}

\begin{frame}\frametitle{MM Example: Maximum Likelihood}
    Maximum likelihood is another special case of a MM estimator \\
    \vspace{2ex}
    We want to maximize the log-likelihood function
    $$\ln L(\theta \mid y_i, x_i) = \sum_{i = 1}^n \ln f(y_i  \mid x_i, \theta)$$
    which gives $K$ first-order conditions (one for each parameter)
    $$\frac{1}{n} \frac{\partial \ln L(\hat{\theta} \mid y_i, x_i)}{\partial \theta} = \frac{1}{n} \sum_{i = 1}^n \frac{\partial \ln f(y_i  \mid x_i, \hat{\theta})}{\partial \theta} = 0$$
    These first-order conditions are the sample analogs of $K$ population moment conditions
    $$E \left[ \frac{\partial \ln f(y_i  \mid x_i, \theta)}{\partial \theta} \right] = 0$$
    so the MLE is also a MM estimator
\end{frame}

\begin{frame}\frametitle{Limitation of the Method of Moments}
    In all of these examples, we have had $K$ moment conditions to estimate $K$ parameters. But what if we have more than $K$ moment conditions? \\
    \vspace{3ex}
    Examples
    \begin{itemize}
    	\item Economic model: Demand parameter appears in demand and supply first-order conditions
    	\item Instruments: Multiple instruments for one endogenous variable
    	\item Model fit: Predicted market shares equal realized market shares
    	\item Statistical assumptions: Errors must be conditional mean-zero and symmetric
    \end{itemize}
    \vspace{3ex}
    The generalized method of moments allows for more moment conditions than parameters
\end{frame}

\begin{frame}\frametitle{}
    \vfill
    \centering
    \begin{beamercolorbox}[center]{title}
        \Large Generalized Method of Moments
    \end{beamercolorbox}
    \vfill
\end{frame}

\begin{frame}\frametitle{Generalized Method of Moments}
    We have $L$ population moment conditions for $K$ parameters (with $L > K$)
    $$E[m(w_i, \theta)] = 0$$
    where $m(\cdot)$ is a vector of $L$ functions, $\theta$ is a vector of $K$ parameters, and $w_i$ is observable data (outcome, explanatory variables, instruments, etc.) \\
    \vspace{2ex}
    We cannot ensure all $L$ moment conditions are solved with only $K$ parameters. Instead, we seek to get as close as possible to solving all $L$ moment conditions by minimizing
    \begin{align*}
    	Q_N(\theta) &= \left[ \frac{1}{n} \sum_{i = 1}^n m(w_i, \theta) \right]' W_N \left[ \frac{1}{n} \sum_{i = 1}^n m(w_i, \theta) \right] \\
    	\Rightarrow \; \hat{\theta} &= \argmin_{\theta} Q_N(\theta)
    \end{align*}
    where $W_N$ is a $L \times L$ positive definite weighting matrix 
\end{frame}

\begin{frame}\frametitle{GMM Example: Instrumental Variables}
    Two-stage least squares is a special case of a GMM estimator \\
    \vspace{2ex}
    Consider the  linear regression model
    $$y_i = \beta' x_i + \varepsilon_i$$
    where $\beta$ is a vector of $K$ parameters. We have a vector of $L$ instruments, $z_i$, that are correlated with $x_i$ and must satisfy $E[\varepsilon_i \mid z_i] = 0$, which (with some math) gives $L$ population moment conditions
    \begin{align*}
    	E[z_i (y_i - \beta' x_i)] &= 0 \\
    	\intertext{Replacing the expectation with the sample analog gives}
    	\frac{1}{n} \sum_{i = 1}^n z_i (y_i - \beta' x_i) &= 0
    \end{align*}
    The GMM estimator, $\hat{\beta}$, is the set of parameters that minimizes
    $$Q_N(\beta) = \left[ \frac{1}{n} \sum_{i = 1}^n z_i (y_i - \beta' x_i) \right]' W_N \left[ \frac{1}{n} \sum_{i = 1}^n z_i (y_i - \beta' x_i) \right]$$
\end{frame}

\begin{frame}\frametitle{Weighting Matrix}
    From the previous example, the GMM estimator, $\hat{\beta}$, is the set of parameters that minimizes
    $$Q_N(\beta) = \left[ \frac{1}{n} \sum_{i = 1}^n z_i (y_i - \beta' x_i) \right]' W_N \left[ \frac{1}{n} \sum_{i = 1}^n z_i (y_i - \beta' x_i) \right]$$
    which depends on the choice of weighting matrix, $W_N$
    \begin{itemize}
    	\item $W_N$ must be a $L \times L$ positive definite matrix
    \end{itemize}
    \vspace{2ex}
    The GMM estimator will be equivalent to the 2SLS estimator if we use
    $$W_N = \sum_{i = 1}^n z_i z_i'$$ \\
    \vspace{2ex}
    But another weighting matrix might yield a ``better'' estimator
\end{frame}

\begin{frame}\frametitle{}
    \vfill
    \centering
    \begin{beamercolorbox}[center]{title}
        \Large Properties of the Generalized Method of Moments Estimator
    \end{beamercolorbox}
    \vfill
\end{frame}

\begin{frame}\frametitle{GMM Assumptions}
    \begin{enumerate}
    	\item The empirical moments obey the law of large numbers
    	$$\frac{1}{n} \sum_{i = 1}^n m_i(\theta_0) \overset{p}{\rightarrow} 0$$
    	\item The empirical moments obey the central limit theorem
    	$$\frac{\sqrt{n}}{n} \sum_{i = 1}^n m_i(\theta_0) \overset{d}{\rightarrow} N(0, S_0)$$
    	where
    	$$S_0 = \plim \frac{1}{n} \sum_{i = 1}^n \sum_{j = 1}^n m_i(\theta_0) m_j(\theta_0)'$$
    	\item The derivatives of the empirical moments converge
    	$$\frac{1}{n} \sum_{i = 1}^n \left( \frac{\partial m(\theta)}{\partial \theta'} \right)_{\theta_0} \overset{p}{\rightarrow} G_0$$
    \end{enumerate}
\end{frame}

\begin{frame}\frametitle{GMM Assumptions}
    \begin{enumerate}
    	\setcounter{enumi}{3}
    	\item The parameters are identified
    	$$\theta_1 \neq \theta_2 \quad \Rightarrow \quad \frac{1}{n} \sum_{i = 1}^n m_i(\theta_1) \neq \frac{1}{n} \sum_{i = 1}^n m_i(\theta_2)$$
    	\item The weighting matrix converges to a finite symmetric positive definite matrix
    	$$W_N \overset{p}{\rightarrow} W_0$$
    \end{enumerate}
\end{frame}

\begin{frame}\frametitle{Consistency of the GMM Estimator}
    The GMM estimator, $\hat{\theta}$, is a consistent estimator of the true parameter values, $\theta_0$
    $$\hat{\theta} \overset{p}{\rightarrow} \theta_0$$
\end{frame}

\begin{frame}\frametitle{Asymptotic Normality of the GMM Estimator}
    The GMM estimator, $\hat{\theta}$, is asymptotically normal with mean $\theta_0$ and known variance
    $$\hat{\theta} \overset{a}{\sim} N\left( \theta_0, \frac{1}{n} (G_0' W_0 G_0)^{-1} (G_0' W_0 S_0 W_0 G_0) (G_0' W_0 G_0)^{-1} \right)$$
    where
    \begin{align*}
    	G_0 &= \plim \frac{1}{n} \sum_{i = 1}^n \left( \frac{\partial m(\theta)}{\partial \theta'} \right)_{\theta_0} \\
    	W_0 &= \plim W_N \\
    	S_0 &= \plim \frac{1}{n} \sum_{i = 1}^n \sum_{j = 1}^n m_i(\theta_0) m_j(\theta_0)'
    \end{align*}
\end{frame}

\begin{frame}\frametitle{}
    \vfill
    \centering
    \begin{beamercolorbox}[center]{title}
        \Large Optimal Generalized Method of Moments
    \end{beamercolorbox}
    \vfill
\end{frame}

\begin{frame}\frametitle{Optimal Weighting Matrix}
    If every weighting matrix yields a GMM estimator that is consistent, then we would ideally like to use the weighting matrix that minimizes the variance of the estimator
    $$Var(\hat{\theta}) = \frac{1}{n} (G_0' W_0 G_0)^{-1} (G_0' W_0 S_0 W_0 G_0) (G_0' W_0 G_0)^{-1}$$
    This variance is minimized when
    \begin{align*}
    	W_0 &= S_0^{-1} = \left[ \plim \frac{1}{n} \sum_{i = 1}^n \sum_{j = 1}^n m_i(\theta_0) m_j(\theta_0)' \right]^{-1} \\
    	& \quad \Rightarrow \quad Var(\hat{\theta}) = \frac{1}{n} (G_0' S_0^{-1} G_0)^{-1}
    \end{align*}
    $S_0$ is the variance-covariance matrix of the empirical moments, so weighting by its inverse minimizes the variance of the GMM estimator
    \begin{itemize}
    	\item This is a generalization of FGLS
    \end{itemize}
\end{frame}

\begin{frame}\frametitle{Optimal (or Two-Step) GMM Estimator}
    To estimate the optimal (or two-step) GMM estimator, perform GMM twice
    \begin{enumerate}
    	\item Perform GMM with any arbitrary weighting matrix
    	\begin{itemize}
    		\item The identity matrix is a simple matrix to use
    		\item Calculate $\widehat{S}$ evaluated at the first-step GMM estimator
    	\end{itemize}
    	\item Perform GMM with $W_N = \widehat{S}^{-1}$ from the first step
    	\begin{itemize}
    		\item Calculate $\widehat{S}$ and $\widehat{G}$ evaluated at the second-step GMM estimator
    		\item Use $\widehat{S}$ and $\widehat{G}$ to estimate the variance covariance matrix
    	\end{itemize}
    \end{enumerate}
    \vspace{2ex}
    \begin{align*}
    	\widehat{Var}(\hat{\theta}) &= \frac{1}{n} \left( \widehat{G}' \widehat{S}^{-1} \widehat{G} \right)^{-1} \\
    	\intertext{where (assuming independence between observations)}
    	\widehat{G} &= \frac{1}{n} \sum_{i =1 }^n \left( \frac{\partial m_i(\theta)}{\partial \theta'} \right)_{\hat{\theta}} \\
    	\widehat{S} &= \frac{1}{n} \sum_{i = 1}^n m_i(\hat{\theta}) m_i(\hat{\theta})'
    \end{align*}
\end{frame}

\begin{frame}\frametitle{Announcements}
	No class on Tuesday! Monday schedule. \\
	\vspace{3ex}
    Reading for next time
    \begin{itemize}
        \item Optional: Crawford and Yurukoglu (2012)
    \end{itemize}
    \vspace{3ex}
    Upcoming
    \begin{itemize}
        \item Problem Set 2 is posted, due October 17
    \end{itemize}
\end{frame}

\end{document}