\documentclass{beamer}
\usetheme{Boadilla}

\makeatother
\setbeamertemplate{footline}
{
    \leavevmode%
    \hbox{%
    \begin{beamercolorbox}[wd=.4\paperwidth,ht=2.25ex,dp=1ex,center]{author in head/foot}%
        \usebeamerfont{author in head/foot}\insertshortauthor
    \end{beamercolorbox}%
    \begin{beamercolorbox}[wd=.55\paperwidth,ht=2.25ex,dp=1ex,center]{title in head/foot}%
        \usebeamerfont{title in head/foot}\insertshorttitle
    \end{beamercolorbox}%
    \begin{beamercolorbox}[wd=.05\paperwidth,ht=2.25ex,dp=1ex,center]{date in head/foot}%
        \insertframenumber{}
    \end{beamercolorbox}}%
    \vskip0pt%
}
\makeatletter
\setbeamertemplate{navigation symbols}{}

\usepackage[T1]{fontenc}
\usepackage{lmodern}
\usepackage{amssymb,amsmath}
\renewcommand{\familydefault}{\sfdefault}

\usepackage{mathtools}
\usepackage{graphicx}
\usepackage{threeparttable}
\usepackage{booktabs}
\usepackage{siunitx}
\sisetup{parse-numbers=false}

\usepackage{forest}
\tikzset{
Above/.style={
  midway,
  above,
  font=\scriptsize,
  text width=1.5cm,
  align=center,
  },
Below/.style={
  midway,
  below,
  font=\scriptsize,
  text width=1.5cm,
  align=center
  }
}

%\setlength{\OuterFrameSep}{-2pt}
%\makeatletter
%\preto{\@verbatim}{\topsep=-10pt \partopsep=-10pt }
%\makeatother

\title[Lecture 13:\ Generalized Extreme Value Models I]{Lecture 13:\ Generalized Extreme Value Models I}
\author[ResEcon 703:\ Advanced Econometrics]{ResEcon 703:\ Topics in Advanced Econometrics}
\date{Matt Woerman\\University of Massachusetts Amherst}

\begin{document}

{\setbeamertemplate{footline}{} 
\begin{frame}[noframenumbering]
    \titlepage
\end{frame}
}

\begin{frame}\frametitle{Agenda}
    Last time
    \begin{itemize}
        \item Generalized Method of Moments Example in R
    \end{itemize}
    \vspace{2ex}
    Today
    \begin{itemize}
    	\item Nested Logit
    	\item Three-Level Nested Logit
    	\item Logit with Overlapping Nests
    	\item Heteroskedastic Logit
    \end{itemize}
    \vspace{2ex}
    Upcoming
    \begin{itemize}
        \item Reading for next time
        \begin{itemize}
            \item Optional: Train et al. (1987)
        \end{itemize}
        \item Problem sets
        \begin{itemize}
            \item Problem Set 3 is posted, due October 31
        \end{itemize}
    \end{itemize}
\end{frame}

\begin{frame}\frametitle{Course Recap}
    Discrete choice models
    \begin{itemize}
    	\item Random utility model
    	\item Logit model
    \end{itemize}
    \vspace{2ex}
    Estimation methods
    \begin{itemize}
    	\item Maximum likelihood estimation
    	\item Nonlinear least squares
    	\item Generalized method of moments
    \end{itemize}
    \vspace{2ex}
    Now that we know these estimation methods, we will jump back to discrete models
    \begin{itemize}
    	\item Correlations between alternatives
    	\item Taste variation due to unobserved factors
    \end{itemize}
\end{frame}

\begin{frame}\frametitle{}
    \vfill
    \centering
    \begin{beamercolorbox}[center]{title}
        \Large Nested Logit
    \end{beamercolorbox}
    \vfill
\end{frame}

\begin{frame}\frametitle{Nested Logit}
    The nested logit model relaxes the (sometimes overly) strong assumption of the logit model
    \begin{itemize}
    	\item Nested logit allows for the unobserved components of utility ($\varepsilon_{nj}$) to be correlated between alternatives for the same decision maker
    \end{itemize}
    \vspace{2ex}
    The nested logit model groups alternatives into ``nests''
    \begin{itemize}
    	\item $Cov(\varepsilon_{nj}, \varepsilon_{nm}) = 0$ if $j$ and $m$ are in different nests
    	\item $Cov(\varepsilon_{nj}, \varepsilon_{nm}) \geq 0$ if $j$ and $m$ are in the same nest
    \end{itemize}
    \vspace{2ex}
    These correlations relax the substitution patterns of the logit model
    \begin{itemize}
    	\item IIA holds for two alternatives within the same nest
    	\item IIA does not necessarily hold for alternatives in different nests
    \end{itemize}
\end{frame}

\begin{frame}\frametitle{Nested Logit Example}
    Commuters in Boston have four alternatives to commute to work
    \begin{columns}
    	\begin{column}{0.5\textwidth}
   			\begin{itemize}
    			\item Drive alone
    			\item Carpool
    			\item Bus
    			\item Train
    		\end{itemize}
    		\vspace{4ex}
		\end{column}
		\begin{column}{0.5\textwidth}
   			\begin{forest}
		    	[Commuter [Car[Drive][Carpool]] [Transit[Bus][Train]]]
		    \end{forest}
		\end{column}
    \end{columns}
    \vspace{2ex}
    Commute travel mode nests
    \begin{itemize}
    	\item Drive alone and carpool might belong in a ``car'' nest
    	\item Bus and train might belong together in a ``public transit'' nest
    \end{itemize}
    \vspace{2ex}
    Can you rationalize any other sets of nests?
    \begin{itemize}
    	\item Choosing nests is more an art than a science
    \end{itemize}
\end{frame}

\begin{frame}\frametitle{Generalized Extreme Value Distribution}
    We partition the $J$ alternatives into $K$ nonoverlapping subsets denoted $B_1, B_2, \ldots, B_K$ and called ``nests'' \\
    \vspace{2ex}
    The utility for each alternative is again $U_{nj} = V_{nj} + \varepsilon_{nj}$ where the vector of unobserved utility, $\varepsilon_n = (\varepsilon_{n1}, \varepsilon_{n2}, \ldots, \varepsilon_{nJ})$, has cumulative distribution
    $$F(\varepsilon_n) = \exp \left( -\sum_{k = 1}^K \left( \sum_{j \in B_k} e^{-\varepsilon_{nj} / \lambda_k} \right)^{\lambda_k} \right)$$
    which is a type of generalized extreme value (GEV) distribution
    \begin{itemize}
    	\item The marginal distribution of $\varepsilon_{nj}$ is extreme value 
    	\item $Cov(\varepsilon_{nj}, \varepsilon_{nm}) = 0$ if $j \in B_k$ and $m \in B_{\ell}$ with $k \neq \ell$
    	\item $Cov(\varepsilon_{nj}, \varepsilon_{nm}) \geq 0$ if $j \in B_k$ and $m \in B_k$
    	\item $\lambda_k$ is a measure of independence in nest $k$
    \end{itemize}
\end{frame}

\begin{frame}\frametitle{Nested Logit Choice Probabilities}
	The general form for the choice probability of alternative $i$ is
	$$P_{ni} = \int_\varepsilon I(\varepsilon_{nj} - \varepsilon_{ni} < V_{ni} - V_{nj} \; \forall j \neq i) f(\varepsilon_n) d\varepsilon_n$$ \\
	\vspace{2ex}
	The distributional assumption of the nested logit model yields a closed-form solution for the choice probability of alternative $i \in B_k$
	$$P_{ni} = \frac{e^{V_{ni} / \lambda_k} \left( \sum_{j \in B_k} e^{V_{nj} / \lambda_k} \right)^{\lambda_k - 1}}{\sum_{\ell = 1}^K \left( \sum_{j \in B_{\ell}} e^{V_{nj} / \lambda_{\ell}} \right)^{\lambda_{\ell}}}$$ \\
	\vspace{2ex}
	Degree of correlation within a nest
	\begin{itemize}
		\item $\lambda_k$ is a measure of independence in nest $k$ that we will estimate
    	\item $1 - \lambda_k$ is an indicator of correlation within nest $k$
    	\item The nested logit model is consistent with the random utility model when $\lambda_k \in (0, 1] \; \forall k$
	\end{itemize}
\end{frame}

\begin{frame}\frametitle{Nested Logit Substitution Patterns}
    The ratio of nested logit choice probabilities for alternative $i \in B_k$ and $m \in B_{\ell}$ is
    \begin{align*}
    	\frac{P_{ni}}{P_{nm}} &= \frac{e^{V_{ni} / \lambda_k} \left( \sum_{j \in B_k} e^{V_{nj} / \lambda_k} \right)^{\lambda_k - 1}}{e^{V_{nm} / \lambda_{\ell}} \left( \sum_{j \in B_{\ell}} e^{V_{nj} / \lambda_{\ell}} \right)^{\lambda_{\ell} - 1}} \\
    	\intertext{When $k = \ell$, this ratio simplifies to}
    	\frac{P_{ni}}{P_{nm}} &= \frac{e^{V_{ni} / \lambda_k}}{e^{V_{nm} / \lambda_{\ell}}}
    \end{align*} \\
    \vspace{2ex}
	The first ratio depends on all alternatives in nests $k$ and $\ell$
	\begin{itemize}
		\item Independence from irrelevant nests (IIN) holds for different nests
	\end{itemize}
	\vspace{2ex}
    This second ratio is irrelevant of all other alternatives 
    \begin{itemize}
    	\item Independence from irrelevant alternatives (IIA) holds within a nest
    \end{itemize}
\end{frame}

\begin{frame}\frametitle{Decomposition of Nested Logit}
    We can decompose the nested logit model into a two-step logit model
    \begin{enumerate}
    	\item The decision maker chooses a nest
    	\item The decision maker chooses an alternative within that nest
    \end{enumerate}
    \vspace{2ex}
    Then the choice probability of alternative $i \in B_k$ is
    $$P_{ni} = P_{nB_k} P_{ni \mid B_k}$$
    where
    \begin{itemize}
    	\item $P_{nB_k}$ is the probability of choosing nest $k$
    	\item $P_{ni \mid B_k}$ is the probability of choosing alternative $i$ conditional on choosing nest $k$
    \end{itemize}
\end{frame}

\begin{frame}\frametitle{Choice Probabilities of Decomposed Nested Logit}
    Express the utility of alternative $j \in B_k$ as
    $$U_{nj} = W_{nk} + Y_{nj} + \varepsilon_{nj}$$
    where
    \begin{itemize}
    	\item $W_{nk}$ depends on characteristics of nest $k$
    	\item $Y_{nj}$ depends on characteristics of alternative $j$
    \end{itemize}
    \vspace{2ex}
    Then the choice probabilities from the previous slide are
    \begin{align*}
    	P_{nB_k} &= \frac{e^{W_{nk} + \lambda_k I_{nk}}}{\sum_{\ell = 1}^K e^{W_{n \ell} + \lambda_{\ell} I_{n \ell}}} \\
    	P_{ni \mid B_k} &= \frac{e^{Y_{ni} / \lambda_k}}{\sum_{j \in B_k} e^{Y_{nj} / \lambda_k}} \\
    	\intertext{where}
    	I_{nk} &= \ln \sum_{j \in B_k} e^{Y_{nj} / \lambda_k}
    \end{align*}
\end{frame}

\begin{frame}\frametitle{Decomposed Nested Logit Intuition}
	Putting everything together
    \begin{align*}
    	P_{ni} &= P_{ni \mid B_k} P_{nB_k} \\
    	P_{nB_k} &= \frac{e^{W_{nk} + \lambda_k I_{nk}}}{\sum_{\ell = 1}^K e^{W_{n \ell} + \lambda_{\ell} I_{n \ell}}} \\
    	P_{ni \mid B_k} &= \frac{e^{Y_{ni} / \lambda_k}}{\sum_{j \in B_k} e^{Y_{nj} / \lambda_k}} \\
    	I_{nk} &= \ln \sum_{j \in B_k} e^{Y_{nj} / \lambda_k}
    \end{align*}
    \begin{itemize}
    	\item $I_{nk}$ is called the inclusive value of nest $k$
    	\item $W_{nk} + \lambda_k I_{nk}$ is the expected utility of nest $k$
    	\item $P_{nB_k}$ is equivalent to the logit choice probability between nests
    	\item $P_{ni \mid B_k}$ is equivalent to the logit choice probability within a nest
    \end{itemize}
\end{frame}

\begin{frame}\frametitle{Nested Logit Elasticities}
    The nested logit model allows for a richer representation of elasticities than does the logit model \\
    \vspace{2ex}
    Own elasticity of alternative $i$ in nest $k$
    \begin{align*}
    	E_{iz_{ni}} &= \beta_z z_{ni} \left( \frac{1}{\lambda_k} - \frac{1 - \lambda_k}{\lambda_k} P_{ni \mid B_k} - P_{ni} \right) \\
    	\intertext{Cross elasticities of alternative $i \in B_k$ with respect to alternative $m \neq i$}
    	E_{iz_{nm}} &= 
    	\begin{cases}
    		-\beta_z z_{nm} P_{nm} \left( 1 + \frac{1 - \lambda_k}{\lambda_k} \frac{1}{P_{nB_k}} \right) &\text{if } m \in B_k \\
    		-\beta_z z_{nm} P_{nm} &\text{if } m \notin B_k
  		\end{cases}
    \end{align*}
\end{frame}

\begin{frame}\frametitle{Nested Logit Estimation}
	Estimation of the nested logit model is similar to the logit model
	\begin{itemize}
		\item We have more complex choice probabilities in our estimation
		\item MLE is the standard method, but you could use NLS or GMM
		\item Log-likelihood function is not globally concave, so numerical optimization can be more difficult than for logit
	\end{itemize}
	\vspace{2ex}
	It is not recommended to estimate the sequential nested logit model of the previous slides
	\begin{itemize}
		\item You have to bootstrap to get the correct variance-covariance matrix
	\end{itemize}
	\vspace{2ex}
	The \texttt{mlogit} package in R has the functionality to estimate a nested logit model
	\begin{itemize}
		\item More on estimating a nested logit model with \texttt{mlogit} next time
	\end{itemize}
\end{frame}

\begin{frame}\frametitle{Nested Logit with Market-Level Data}
    The nested logit model can also be estimated from market-level data
    \begin{itemize}
    	\item You observe the price, market share, and characteristics of every cereal brand at the grocery store, and you want to estimate the structural parameters of consumer decision making that explain those purchases
    \end{itemize}
    \vspace{1ex}
    When aggregated over many consumers, choice probabilities become market shares
    $$S_i = \frac{e^{V_{i} / \lambda_k} \left( \sum_{j \in B_k} e^{V_{j} / \lambda_k} \right)^{\lambda_k - 1}}{\sum_{\ell = 1}^K \left( \sum_{j \in B_{\ell}} e^{V_{j} / \lambda_{\ell}} \right)^{\lambda_{\ell}}}$$
    If we assume representative utility is liner, $V_j = \beta' x_j$
    $$\ln(S_i) - \ln(S_m) = \beta' (x_i - x_j) + (1 - \lambda_k) \ln S_{i \mid B_k} - (1 - \lambda_{\ell}) S_{m \mid B_{\ell}}$$
    Set one alternative to be your reference in its own nest (usually the outside option) and estimate the linear regression for the other alternatives
    $$\ln(S_i) - \ln(S_0) = \beta' (x_i - x_0) + (1 - \lambda_k) \ln (S_{i \mid B_k}) + \omega_i$$
\end{frame}

\begin{frame}\frametitle{}
    \vfill
    \centering
    \begin{beamercolorbox}[center]{title}
        \Large Three-Level Nested Logit
    \end{beamercolorbox}
    \vfill
\end{frame}

\begin{frame}\frametitle{Three-Level Nested Logit}
    We can model a richer set of correlations between alternatives by including multiple levels of nests and subnests
    \begin{itemize}
    	\item As in the (two-level) nested logit model, we first group alternatives into nests
    	\item Then, within each nest, we further group alternatives into subnests
    \end{itemize}
    \vspace{2ex}
    Correlations within nests and subnests
    \begin{itemize}
    	\item $\lambda_k$ defines the level of independence within nest $k$
    	\item $\sigma_{mk}$ defines the level of independence within subnest $m$ in nest $k$
    	\item $1 - \lambda_k \sigma_{mk}$ is an indicator of correlation within subnest $m$ in nest $k$
    	\item This model is consistent with the random utility model when $\lambda_k \in (0, 1] \; \forall k$ and $\sigma_{mk} \in (0, 1] \; \forall m$
    \end{itemize}
    \vspace{2ex}
    We can even model more than three levels of nests
    \begin{itemize}
    	\item There are also other ways to model these more complex correlation structures\ldots
    \end{itemize}
\end{frame}

\begin{frame}\frametitle{Three-Level Nested Logit Example}
    Apartments in Amherst and Northampton \\
    \vspace{1ex}
    \centering
    \begin{forest}
    	[Renter [Amherst [1-Bedroom[A][B][C]] [2-Bedroom[D][E][F]]] [Northampton [1-Bedroom[G][H][I]] [2-Bedroom[J][K][L]]]]
    \end{forest}
    \begin{itemize}
    	\item All 1-bedroom apartments in Amherst are correlated
    	\item 1-bedroom apartments in Amherst are also (less) correlated with 2-bedroom apartments in Amherst
    	\item 1-bedroom apartments in Amherst are uncorrelated with any apartment in Northampton
    \end{itemize}
\end{frame}

\begin{frame}\frametitle{Three-Level Nested Logit Example}
    Or maybe the nesting should be in the other order? \\
    \vspace{1ex}
    \centering
    \begin{forest}
    	[Renter [1-Bedroom [Amherst[A][B][C]] [Northampton[G][H][I]]] [2-Bedroom [Amherst[D][E][F]] [Northampton[J][K][L]]]]
    \end{forest}
    \begin{itemize}
    	\item All 1-bedroom apartments in Amherst are correlated
    	\item 1-bedroom apartments in Amherst are also (less) correlated with 1-bedroom apartments in Northampton
    	\item 1-bedroom apartments in Amherst are uncorrelated with any 2-bedroom apartment
    \end{itemize}
\end{frame}

\begin{frame}\frametitle{}
    \vfill
    \centering
    \begin{beamercolorbox}[center]{title}
        \Large Logit with Overlapping Nests
    \end{beamercolorbox}
    \vfill
\end{frame}

\begin{frame}\frametitle{Logit with Overlapping Nests}
    The nested logit model requires that nests (and subnests) are exhaustive and mutually exclusive
    \begin{itemize}
    	\item We have to decide which alternatives belong in nests (or subnests) together
    	\item This structure is more flexible than the logit model
    	\item But it still requires us to make assumptions about which alternatives might be correlated and which are assumed to be uncorrelated
    \end{itemize}
    \vspace{3ex}
    Is there a way to represent correlations even more flexibly?
    \begin{itemize}
    	\item Yes, if we allow nests to overlap
    	\item At the extreme, we can let every pair of alternatives form its own nest
    \end{itemize}
\end{frame}

\begin{frame}\frametitle{Paired Combinatorial Logit}
    Each pair of alternatives is a nest (note that this does not comply with our prior definition of a ``nest'')
    \begin{itemize}
    	\item We have $J (J - 1) / 2$  nests
    	\item Each nests has only two alternatives
    	\item Each alternative is in $J - 1$ nests
    \end{itemize}
    \vspace{2ex}
    Correlation between alternatives $i$ and $j$
    \begin{itemize}
    	\item $\lambda_{ij}$ defines the independence between alternatives $i$ and $j$
    	\item $1 - \lambda_{ij}$ is an indicator of the correlation between alternatives $i$ and $j$
    	\item We have $J (J - 1) / 2$ $\lambda$ parameters to estimate
    	\begin{itemize}
    		\item We can only identify $J (J - 1) / 2 - 1$ covariance parameters, so we have to normalize at least one $\lambda = 1$
    	\end{itemize}
    	\item This model is consistent with the random utility model when $\lambda_{ij} \in (0, 1] \; \forall i,j$ 
    \end{itemize}
\end{frame}

\begin{frame}\frametitle{Paired Combinatorial Logit Choice Probabilities}
    The choice probability for the paired combinatorial logit model is
    $$P_{ni} = \frac{\sum_{j \neq i} e^{V_{ni} / \lambda_{ij}} \left( e^{V_{ni} / \lambda_{ij}} + e^{V_{nj} / \lambda_{ij}} \right)^{\lambda_{ij} - 1}}{\sum_{k = 1}^{J - 1} \sum_{\ell = k + 1}^J \left( e^{V_{nk} / \lambda_{k \ell}} + e^{V_{n \ell} / \lambda_{k \ell}} \right)^{\lambda_{k \ell}}}$$ \\
    \vspace{2ex}
    The denominator is analogous to the nested logit denominator
    \begin{itemize}
    	\item Sum over all nests
    	\item Within a nest, sum over all elements within the nest
    \end{itemize}
    \vspace{2ex}
    The numerator is analogous to the nested logit numerator
    \begin{itemize}
    	\item We now have to sum over all nests that contain alternative $i$
    \end{itemize}
\end{frame}

\begin{frame}\frametitle{}
    \vfill
    \centering
    \begin{beamercolorbox}[center]{title}
        \Large Heteroskedastic Logit
    \end{beamercolorbox}
    \vfill
\end{frame}

\begin{frame}\frametitle{Heteroskedastic Logit}
    We can also use the GEV distributional assumption to allow for heteroskedasticity of alternatives
    \begin{itemize}
    	\item The variance of unobserved utility can be different for each alternative
    \end{itemize}
    \vspace{2ex}
    Utility is specified as $U_{nj} = V_{nj} + \varepsilon_{nj}$ with
    $$Var(\varepsilon_{nj}) = \frac{(\theta_j \pi)^2}{6}$$
    \begin{itemize}
    	\item We have $J$ variance parameters to estimate
    	\item We can only identify $J - 1$, so we have to normalize at least one $\theta = 1$
    \end{itemize}
    \vspace{2ex}
    The choice probabilities for heteroskedastic logit do not have a closed-form expression
    \begin{itemize}
    	\item We have to use simulation methods to get choice probabilities and subsequently estimate model parameters
    \end{itemize}
\end{frame}

\begin{frame}\frametitle{Announcements}
    Reading for next time
    \begin{itemize}
        \item Optional: Train et al. (1987)
    \end{itemize}
    \vspace{3ex}
    Office hours
    \begin{itemize}
    	\item Reminder: 2:00--3:00 on Tuesdays in 218 Stockbridge
    \end{itemize}
    \vspace{3ex}
    Upcoming
    \begin{itemize}
        \item Problem Set 3 is posted, due October 31
    \end{itemize}
\end{frame}

\end{document}