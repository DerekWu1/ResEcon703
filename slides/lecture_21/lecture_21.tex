\documentclass{beamer}
\usetheme{Boadilla}

\makeatother
\setbeamertemplate{footline}
{
    \leavevmode%
    \hbox{%
    \begin{beamercolorbox}[wd=.4\paperwidth,ht=2.25ex,dp=1ex,center]{author in head/foot}%
        \usebeamerfont{author in head/foot}\insertshortauthor
    \end{beamercolorbox}%
    \begin{beamercolorbox}[wd=.55\paperwidth,ht=2.25ex,dp=1ex,center]{title in head/foot}%
        \usebeamerfont{title in head/foot}\insertshorttitle
    \end{beamercolorbox}%
    \begin{beamercolorbox}[wd=.05\paperwidth,ht=2.25ex,dp=1ex,center]{date in head/foot}%
        \insertframenumber{}
    \end{beamercolorbox}}%
    \vskip0pt%
}
\makeatletter
\setbeamertemplate{navigation symbols}{}

\usepackage[T1]{fontenc}
\usepackage{lmodern}
\usepackage{amssymb,amsmath}
\renewcommand{\familydefault}{\sfdefault}

\DeclareMathOperator*{\argmax}{argmax}

\usepackage{mathtools}
\usepackage{graphicx}
\usepackage{threeparttable}
\usepackage{booktabs}
\usepackage{siunitx}
\sisetup{parse-numbers=false}

%\setlength{\OuterFrameSep}{-2pt}
%\makeatletter
%\preto{\@verbatim}{\topsep=-10pt \partopsep=-10pt }
%\makeatother

\title[Lecture 21:\ Dynamic Discrete Choice I]{Lecture 21:\ Dynamic Discrete Choice I}
\author[ResEcon 703:\ Advanced Econometrics]{ResEcon 703:\ Topics in Advanced Econometrics}
\date{Matt Woerman\\University of Massachusetts Amherst}

\begin{document}

{\setbeamertemplate{footline}{} 
\begin{frame}[noframenumbering]
    \titlepage
\end{frame}
}

\begin{frame}\frametitle{Agenda}
    Last time
    \begin{itemize}
        \item Individual-Specific Parameters Example in R
    \end{itemize}
    \vspace{2ex}
    Today
    \begin{itemize}
    	\item Static Models with Panel Data
    	\item Dynamic Discrete Choice Examples
        \item Dynamic Discrete Choice 
    \end{itemize}
    \vspace{2ex}
    Upcoming
    \begin{itemize}
        \item Reading for next time
        \begin{itemize}
            \item Rust (1987)
        \end{itemize}
        \item Problem sets
        \begin{itemize}
            \item Problem Set 4 is posted, due November 21
        \end{itemize}
    \end{itemize}
\end{frame}

\begin{frame}\frametitle{Static Vs.\ Dynamic Models}
    Static structural models
    \begin{itemize}
    	\item An agent maximizes their objective function within the current time period without considering the effect on future time periods
    	\item Some simple ``dynamics'' can be incorporated by including
    	\begin{itemize}
    		\item Lagged or future explanatory variables
    		\item State dependence through lagged outcome variables
    	\end{itemize}
    	\item All the models we have discussed have been static
    	\item Panel data models can be (and usually are) static
    \end{itemize}
    \vspace{2ex}
    Dynamic structural models
    \begin{itemize}
    	\item An agent maximizes their objective function while explicitly considering the effect on future time periods
    	\begin{itemize}
    		\item A choice in one period may change the choice set in future periods
    		\item A choice in one period may change the utility of future choices
    	\end{itemize}
    	\item Modeling this dynamic behavior requires a more complex framework
    \end{itemize}
\end{frame}

\begin{frame}\frametitle{}
    \vfill
    \centering
    \begin{beamercolorbox}[center]{title}
        \Large Static Models with Panel Data
    \end{beamercolorbox}
    \vfill
\end{frame}

\begin{frame}\frametitle{Logit Model with Panel Data}
    We can use the logit model to model discrete choices with panel data
    $$U_{njt} = \beta' x_{njt} + \varepsilon_{njt} \quad \Rightarrow \quad P_{nit} = \frac{e^{\beta' x_{nit}}}{\sum_j e^{\beta' x_{njt}}}$$ \\
    \vspace{2ex}
    The logit assumption has to hold
    $$\varepsilon_{njt} \sim \text{i.i.d.\ type I extreme value (Gumbel) with } Var(\varepsilon_{njt}) = \frac{\pi^2}{6}$$ \\
    \begin{itemize}
        \item But the unobserved characteristics of a decision maker that affect choice are unlikely to be independent over time
    \end{itemize}
    \vspace{2ex}
    The logit model with panel data is a sequence of static choices, not a fully dynamic model
    \begin{itemize}
        \item We assume the decision maker maximizes utility in each time period
        \item But we do not represent how a decision can affect future choices
    \end{itemize}
\end{frame}

\begin{frame}\frametitle{Mixed Logit Model with Panel Data}
    We can better represent a sequence of choices over time using the mixed logit model
    $$U_{njt} = \beta_n' x_{njt} + \varepsilon_{njt} \quad \Rightarrow \quad P_{ni} = \int \prod_{t = 1}^T \frac{e^{\beta' x_{ni_{t}t}}}{\sum_{j = 1}^J e^{\beta' x_{nj_{t}t}}} f(\beta \mid \theta) d \beta$$ \\
    \vspace{2ex}
    The individual-specific coefficients represent unobserved preferences
    \begin{itemize}
        \item We model these coefficients as random coefficients and estimate their distributions
        \item These individual coefficients yield unobserved correlations in choices over time periods
    \end{itemize}
    \vspace{2ex}
    Although the mixed logit model better represents choices over multiple time periods, it is still inherently a sequence of static choices
\end{frame}

\begin{frame}\frametitle{Simple ``Dynamics'' in Static Models}
    When using these static models, we can include lagged outcome variables to model
    \begin{itemize}
        \item Habit formation
        \item Variety-seeking behavior
        \item Switching costs
        \item Brand loyalty
    \end{itemize}
    \vspace{2ex}
    These are all examples of how past choices affect the decision maker's utility in the current time period \\
    \vspace{2ex}
    But if past choices affect utility in the current time period, then the current choice affects future utility
    \begin{itemize}
        \item A rational decision maker will consider these effects on future utility when making the current choice
        \item To fully capture this discrete choice framework, we need to use a dynamic discrete choice model
    \end{itemize}
\end{frame}

\begin{frame}\frametitle{}
    \vfill
    \centering
    \begin{beamercolorbox}[center]{title}
        \Large Dynamic Discrete Choice Examples
    \end{beamercolorbox}
    \vfill
\end{frame}

\begin{frame}\frametitle{Dynamic Discrete Choice Example}
    Why do people attend college? (Or graduate school?)
    \begin{itemize}
        \item Because four years in college provides more utility than anything else the individual could have done in those four years?
        \item Or because college opens up a new set of jobs and higher salaries compared to not attending college?
    \end{itemize}
    \vspace{2ex}
    A static model implicitly assumes that the first answer is correct
    \begin{itemize}
        \item The static model has no way to represent that the future choice set and salaries will be different after attending to college
    \end{itemize}
    \vspace{2ex}
    To consider the second answer, we need to use a dynamic discrete choice model
    \begin{itemize}
        \item This model will explicitly represent how the decision to attend college affects the future choice set and salaries
    \end{itemize}
\end{frame}

\begin{frame}\frametitle{Decision Maker with Two-Period Dynamics}
    A decision maker thinking about college considers two time periods
    \begin{enumerate}
        \item College ($C$) or work ($W$) for four years
        \begin{itemize}
            \item $U_{1C}$: utility in period 1 from four years in college
            \item $U_{1W}$: utility in period 1 from four years working
        \end{itemize}
        \item A set of $J$ possible jobs for a career over many future years
        \begin{itemize}
            \item $U_{2j}^C$: utility in period 2 from job $j$ after attending college
            \item $U_{2j}^W$: utility in period 2 from job $j$ after working
        \end{itemize}
    \end{enumerate}
    \vspace{2ex}
    The total utility obtained from attending college or working in period 1 is
    \begin{align*}
        TU_C &= U_{1C} + \lambda \max_j(U_{2j}^C) \\
        TU_W &= U_{1W} + \lambda \max_j(U_{2j}^W)
    \end{align*}
    where $\lambda$ reflects the relative weighting of the two periods \\
    \vspace{2ex}
    The decision maker attends college if and only if $TU_C > TU_W$
\end{frame}

\begin{frame}\frametitle{Econometrician with Two-Period Dynamics}
    We decompose the utility of each alternative into an observed and an unobserved component
    \begin{alignat*}{3}
        U_{1C} &= V_{1C} + \varepsilon_{1C} \qquad && U_{2j}^C &&= V_{2j}^C + \varepsilon_{2j}^C \\
        U_{1W} &= V_{1W} + \varepsilon_{1W} \qquad && U_{2j}^W &&= V_{2j}^W + \varepsilon_{2j}^W
    \end{alignat*}
    and define $\varepsilon = \{\varepsilon_{1C},\varepsilon_{1W}, \varepsilon_{2j}^C, \varepsilon_{2j}^W\}$ with joint density $f(\varepsilon)$ \\
    \vspace{2ex}
    The probability that the decision maker chooses to attend college is
    \begin{align*}
        P_C &= \Pr(TU_C > TU_W) \\
        &= \int I \left[ V_{1C} + \varepsilon_{1C} + \lambda \max_j(V_{2j}^C + \varepsilon_{2j}^C) \right. \\
        &\left. \qquad \qquad > V_{1W} + \varepsilon_{1W} + \lambda \max_j(V_{2j}^W + \varepsilon_{2j}^W)\right] f(\varepsilon) d \varepsilon
    \end{align*}
    We have to approximate this integral using simulation
\end{frame}

\begin{frame}\frametitle{Simplifications for Two-Period Dynamics}
    If we assume that unobserved utility in the first period, $\varepsilon_{1C}$ and $\varepsilon_{1W}$, is i.i.d.\ extreme value, the choice probability of attending college simplifies to
    $$P_C = \int \frac{e^{V_{1C} + \lambda \max_j(V_{2j}^C + \varepsilon_{2j}^C)}}{e^{V_{1C} + \lambda \max_j(V_{2j}^C + \varepsilon_{2j}^C)} + e^{V_{1W} + \lambda \max_j(V_{2j}^W + \varepsilon_{2j}^W)}} g(\varepsilon_2) d \varepsilon_2$$
    which is a simpler integral to simulate \\
    \vspace{3ex}
    If we also assume that unobserved utility in the second period, $\varepsilon_{2j}^C$ and $\varepsilon_{2j}^W$, is i.i.d.\ extreme value, we get the joint probability of attending college in period 1 and then taking job $i$ in period 2 as
    $$P_{Ci} = P_C \times \frac{e^{V_{2i}^C}}{\sum_{j = 1}^J e^{{V_{2j}^C}}}$$
    where $P_C$ is the probability of attending college given above
\end{frame}

\begin{frame}\frametitle{Example With Three Periods}
    After many years working job $j$, the decision maker reaches retirement age and has a new choice to make
    \begin{itemize}
    	\item Continue working full time
    	\item Work part time and spend retirement funds
    	\item Retire and collect social security and/or pension funds
    \end{itemize}
    \vspace{2ex}
    The retirement plan, social security payout, etc.\ differs for each of the possible jobs in period 2
    \begin{itemize}
    	\item In period 2, the decision maker will consider how their job will affect utility in period 3
    	\item In period 1, the decision maker will consider how the decision to attend college will affect all periods
    \end{itemize}
\end{frame}

\begin{frame}\frametitle{Decision Maker with Three-Period Dynamics}
    A decision maker thinking about college considers three time periods
    \begin{enumerate}
        \item College ($C$) or work ($W$) for four years
        \item A set of $J$ possible jobs for a career over many future years
        \item A set of $S$ possible retirement plans
        \begin{itemize}
        	\item $U_{3s}^{Cj}$: utility in period 3 from retirement plan $s$ after attending college in period 1 and working job $j$ in period 2
            \item $U_{3s}^{Wj}$: utility in period 3 from retirement plan $s$ after working in period 1 and working job $j$ in period 2
        \end{itemize}
    \end{enumerate}
    \vspace{1ex}
    The total utility obtained from attending college or working in period 1 is
    \begin{align*}
        TU_C &= U_{1C} + \lambda \max_j \left[ U_{2j}^C + \theta \max_s (U_{3s}^{Cj}) \right] \\
        TU_W &= U_{1W} + \lambda \max_j \left[ U_{2j}^W + \theta \max_s (U_{3s}^{Wj}) \right]
    \end{align*}
    where $\lambda$ and $\theta$ reflect the relative weighting of the three periods \\
    \vspace{1ex}
    The decision maker goes to college if and only if $TU_C > TU_W$
\end{frame}

\begin{frame}\frametitle{Econometrician with Three-Period Dynamics}
    We decompose the utility of each alternative into an observed and an unobserved component
    \begin{alignat*}{5}
        U_{1C} &= V_{1C} + \varepsilon_{1C} \qquad && U_{2j}^C &&= V_{2j}^C + \varepsilon_{2j}^C \qquad && U_{3s}^{Cj} &&= V_{3s}^{Cj} + \varepsilon_{3s}^{Cj} \\
        U_{1W} &= V_{1W} + \varepsilon_{1W} \qquad && U_{2j}^W &&= V_{2j}^W + \varepsilon_{2j}^W \qquad && U_{3s}^{Wj} &&= V_{3s}^{Wj} + \varepsilon_{3s}^{Wj} 
    \end{alignat*}
    and define $\varepsilon = \{\varepsilon_{1C},\varepsilon_{1W}, \varepsilon_{2j}^C, \varepsilon_{2j}^W, \varepsilon_{3s}^{Cj}, \varepsilon_{3s}^{Wj}\}$ with joint density $f(\varepsilon)$ \\
    \vspace{3ex}
    The probability that the decision maker chooses to attend college is
    \begin{align*}
        P_C &= \Pr(TU_C > TU_W) \\
        &= \int I \left\{ V_{1C} + \varepsilon_{1C} + \lambda \max_j \left[ V_{2j}^C + \varepsilon_{2j}^C + \theta \max_s (V_{3s}^{Cj} + \varepsilon_{3s}^{Cj}) \right] \right. \\
        & \left. \qquad \qquad > V_{1W} + \varepsilon_{1W} + \lambda \max_j \left[ V_{2j}^W + \varepsilon_{2j}^W + \theta \max_s (V_{3s}^{Wj} + \varepsilon_{3s}^{Wj}) \right] \right\} \\
        & \qquad \times f(\varepsilon) d \varepsilon
    \end{align*}
\end{frame}

\begin{frame}\frametitle{}
    \vfill
    \centering
    \begin{beamercolorbox}[center]{title}
        \Large Dynamic Discrete Choice
    \end{beamercolorbox}
    \vfill
\end{frame}

\begin{frame}\frametitle{Dynamic Optimization Notation and Terminology}
    Some common notation and terminology for dynamic structural models
    \begin{itemize}
        \item $\{i_1, i_2, \ldots, i_t\}$: sequence of choices up to and including period $t$
        \item $U_{tj}(i_1, i_2, \ldots, i_{t-1})$: utility obtained in period $t$ from alternative $j$, which depends on all previous choices
        \item $TU_{tj}(i_1, i_2, \ldots, i_{t-1})$: total utility (current and all future time periods) obtained from choosing alternative $j$ in period $t$
        \begin{itemize}
            \item Known as the ``conditional value function''
        \end{itemize}
        \item $TU_t(i_1, i_2, \ldots, i_{t-1})$: total utility obtained from the optimal choice in period $t$
        \begin{itemize}
            \item $TU_t(i_1, i_2, \ldots, i_{t-1}) = \max_j TU_{tj}(i_1, i_2, \ldots, i_{t-1})$
            \item Known as the ``value function'' or ``valuation function'' at time $t$
        \end{itemize}
    \end{itemize}
    \vspace{2ex}
    We need to calculate all possible $TU_{tj}(i_1, \ldots, i_{t-1})$ in order to express the optimal choice in each time period
\end{frame}

\begin{frame}\frametitle{Bellman Equation for Dynamic Discrete Choice}
    The decision maker chooses optimally (maximizes utility) in the current period knowing they will also choose optimally in every future period (and discounting the future with discount rate $\delta$), which yields an expression for the value function at time $t$
    $$TU_t(i_1, \ldots, i_{t-1}) = \max_j \left[ U_{tj}(i_1, \ldots, i_{t-1}) + \delta TU_{t+1}(i_1, \ldots, i_t = j) \right]$$
    This relation is the Bellman equation for dynamic discrete choice \\
    \vspace{2ex}
    We can also write down a Bellman equation for the conditional valuation function, $TU_{tj}(i_1, \ldots, i_{t-1})$
    $$TU_{tj}(i_1, \ldots, i_{t-1}) = U_{tj}(i_1, \ldots, i_{t-1}) + \delta \max_k \left[ TU_{t+1,k}(i_1, \ldots, i_t = j) \right]$$
\end{frame}

\begin{frame}\frametitle{Applying the Bellman Equation}
    If the number of time periods is finite, we can apply the Bellman equation through backward recursion to calculate all possible $TU_{tj}(i_1, \ldots, i_{t-1})$
    \begin{enumerate}
        \item Start in the last time period, $t = T$, with
        $$TU_{Tj}(i_1, \ldots, i_{T-1}) = U_{Tj}(i_1, \ldots, i_{T-1})$$
        \item Calculate total utility in period $T - 1$, $TU_{T-1,j}(i_1, \ldots, i_{T-2})$, as a function of the values of $TU_{Tj}(i_1, \ldots, i_{T-1})$ from step 1
        \item Continue working backward until you reach period 1
    \end{enumerate}
    \vspace{2ex}
    We have to calculate $U_{tj}(i_1, \ldots, i_{t-1})$ for each $t$, each $j$, and each $\{i_1, i_2, \ldots, i_{t-1}\}$
    \begin{itemize}
        \item If there are $J$ alternatives in each of $T$ time periods, we have to calculate $(J^T)T$ utilities
        \item This computational burden is known as the ``curse of dimensionality''
    \end{itemize}
\end{frame}

\begin{frame}\frametitle{Choice Probabilities in Dynamic Discrete Choice}
    We will ultimately use these conditional value functions, $TU_{tj}(i_1, \ldots, i_{t-1})$, to crate an expression for choice probabilities \\
    \vspace{2ex}
    For example, the probability of choosing alternative $i$ in period 1 is
    \begin{align*}
        P_{1i} &= \Pr(TU_{1i} > TU_{1j} \; \forall j \neq i) \\
        &= \int I \left[ TU_{1i}(\varepsilon) > TU_{1j}(\varepsilon) \; \forall j \neq i \right] f(\varepsilon) d \varepsilon
    \end{align*} \\
    \vspace{2ex}
    We have to simulate this choice probability by taking random draws from an assumed joint density of all unobserved utilities, $f(\varepsilon)$ \\
    \vspace{2ex}
    See Chapter 7.7.3 in the Train textbook for more details
\end{frame}

\begin{frame}\frametitle{Uncertainty in Dynamic Discrete Choice Models}
    So far we have assumed that the decision maker has perfect information about the future
    \begin{itemize}
        \item Utility of each alternative in each future time period
        \item How every possible sequence of choices affects this future utility
        \item But this is unlikely to be true!
    \end{itemize}
    \vspace{2ex}
    We can model utility as being a function factors that are unknown in previous periods
    \begin{itemize}
        \item The decision maker maximizes total expected utility with the expectation taken over the density of the unknown factors
        \item This expectation adds another integral that has to be simulated, adding yet another layer of complexity and dimensionality to a problem that already suffers from the curse of dimensionality
    \end{itemize}
    \vspace{2ex}
    See Chapter 7.7.3 in the Train textbook for more details
\end{frame}

\begin{frame}\frametitle{Simplifications for Dynamic Discrete Choice Models}
    Two ways to simplify dynamic discrete choice models
    \begin{enumerate}
        \item Use as few time periods as possible
        \begin{itemize}
            \item We could model the college-job-retirement sequence of choices annually (or monthly, weekly, daily) instead of three broad time periods
            \item Estimation is feasible with three time periods, but it becomes (at least) an order of magnitude more difficult with 60 individual year
        \end{itemize}
        \item Assume the factors that the decision maker does not observe are the same factors that the econometrician does not observe, and these factors are i.i.d.\ extreme value
        \begin{itemize}
            \item Choice probabilities have closed-form expressions that are easy to calculate
            \item This assumption is unrealistic, but it may be the only way to make the model tractable
        \end{itemize}
    \end{enumerate}
    \vspace{2ex}
    See Chapter 7.7.3 in the Train textbook for more details
\end{frame}

\begin{frame}\frametitle{Announcements}
    Reading for next time
    \begin{itemize}
        \item Rust (1987)
    \end{itemize}
    \vspace{3ex}
    Office hours
    \begin{itemize}
    	\item Reminder: 2:00--3:00 on Tuesdays in 218 Stockbridge
    \end{itemize}
    \vspace{3ex}
    Upcoming
    \begin{itemize}
        \item Problem Set 4 is posted, due November 21
    \end{itemize}
\end{frame}

\end{document}