\documentclass{beamer}
\usetheme{Boadilla}

\makeatother
\setbeamertemplate{footline}
{
    \leavevmode%
    \hbox{%
    \begin{beamercolorbox}[wd=.4\paperwidth,ht=2.25ex,dp=1ex,center]{author in head/foot}%
        \usebeamerfont{author in head/foot}\insertshortauthor
    \end{beamercolorbox}%
    \begin{beamercolorbox}[wd=.55\paperwidth,ht=2.25ex,dp=1ex,center]{title in head/foot}%
        \usebeamerfont{title in head/foot}\insertshorttitle
    \end{beamercolorbox}%
    \begin{beamercolorbox}[wd=.05\paperwidth,ht=2.25ex,dp=1ex,center]{date in head/foot}%
        \insertframenumber{}
    \end{beamercolorbox}}%
    \vskip0pt%
}
\makeatletter
\setbeamertemplate{navigation symbols}{}

\usepackage[T1]{fontenc}
\usepackage{lmodern}
\usepackage{amssymb,amsmath}
\renewcommand{\familydefault}{\sfdefault}

\usepackage{mathtools}
\usepackage{graphicx}
\usepackage{threeparttable}
\usepackage{booktabs}
\usepackage{siunitx}
\sisetup{parse-numbers=false}

%\setlength{\OuterFrameSep}{-2pt}
%\makeatletter
%\preto{\@verbatim}{\topsep=-10pt \partopsep=-10pt }
%\makeatother

\title[Lecture 12:\ Generalized Method of Moments II]{Lecture 12:\ Generalized Method of Moments II}
\author[ResEcon 703:\ Advanced Econometrics]{ResEcon 703:\ Topics in Advanced Econometrics}
\date{Matt Woerman\\University of Massachusetts Amherst}

\begin{document}

{\setbeamertemplate{footline}{} 
\begin{frame}[noframenumbering]
    \titlepage
\end{frame}
}

\begin{frame}\frametitle{Agenda}
    Last time
    \begin{itemize}
        \item Generalized Method of Moments
    \end{itemize}
    \vspace{2ex}
    Today
    \begin{itemize}
        \item Generalized Method of Moments Example in R
    \end{itemize}
    \vspace{2ex}
    Upcoming
    \begin{itemize}
        \item Reading for next time
        \begin{itemize}
            \item Train textbook, Chapter 4
        \end{itemize}
        \item Problem sets
        \begin{itemize}
            \item Problem Set 2 is due
            \item Problem Set 3 will be posted soon, due October 31
        \end{itemize}
    \end{itemize}
\end{frame}

\begin{frame}\frametitle{Generalized Method of Moments}
    We have population moment conditions
    \begin{align*}
        E[m(w_i, \theta_0)] &= 0 \\
        \intertext{The GMM estimator uses the sample analogs of these moment conditions}
        \frac{1}{n} \sum_{i = 1}^n m(w_i, \theta) & \\
        \intertext{The GMM estimator is the set of parameters that ``solves''}
        \frac{1}{n} \sum_{i = 1}^n m(w_i, \hat{\theta}) &= 0
    \end{align*} \\
    \vspace{2ex}
    The GMM estimator is more flexible and less parametric (semi-parametric?) than MLE or NLS
\end{frame}

\begin{frame}\frametitle{}
    \vfill
    \centering
    \begin{beamercolorbox}[center]{title}
        \Large Generalized Method of Moments Example in R
    \end{beamercolorbox}
    \vfill
\end{frame}

\begin{frame}\frametitle{Binary Logit Model Example}
    We are studying how consumers make choices about expensive and highly energy-consuming appliances in their homes. We have data on 600 households who rent a studio apartment and whether or not they choose to purchase a window air conditioning unit. For each household, we observe the purchase price of the air conditioner, its annual operating cost, and some household-level data. (To simplify things, we assume there is only one ``representative'' air conditioner for each household.)
    \vspace{3ex}
    \begin{columns}
        \begin{column}{0.5\textwidth}
            Alternative-specific data
            \begin{itemize}
                \item cost\_system: system cost (price)
                \item cost\_operating: annual operating cost
            \end{itemize}
        \end{column}
        \begin{column}{0.5\textwidth}
            Household data
            \begin{itemize}
                \item elec\_price: electricity price
                \item square\_feet: apartment square footage
            \end{itemize}
            \vspace{2ex}
        \end{column}
    \end{columns}
\end{frame}

\begin{frame}[fragile]\frametitle{Load Dataset}
    <<R CODE HERE>>
\end{frame}

\begin{frame}[fragile]\frametitle{Dataset}
    <<R CODE HERE>>
\end{frame}

\begin{frame}\frametitle{Binary Logit Model to Estimate with GMM}
    The representative utility from having the window air conditioner is (we implicitly normalize the utility of no air conditioner to be zero)
    $$V_n = \beta_0 + \beta_1 SC_n + \beta_2 OC_n$$ \\
    \vspace{2ex}
    From the logit model, the probability of purchasing the air conditioner is
    \begin{align*}
        P_n &= \frac{e^{\beta_0 + \beta_1 SC_n + \beta_2 OC_n}}{1 + e^{\beta_0 + \beta_1 SC_n + \beta_2 OC_n}} \\
        &= \frac{1}{1 + e^{-(\beta_0 + \beta_1 SC_n + \beta_2 OC_n)}}
    \end{align*} \\
    \vspace{2ex}
    We will estimate this model using the moments we derived from the MLE first-order conditions (see previous slides or Train textbook)
    \begin{align*}
        &E[[y_n - P_n(x_n, \beta_0)] x_n] = 0 \\
        \Rightarrow \quad &\frac{1}{N} \sum_{n = 1}^N [y_n - P_n(x_n, \hat{\beta})] x_n = 0
    \end{align*}
\end{frame}

\begin{frame}[fragile]\frametitle{Generalized Method of Moments in R}
    <<R CODE HERE>>
    \texttt{gmm()} requires that you create a function, \texttt{g}, that
    \begin{enumerate}
        \item Takes a set of parameters and a data \emph{matrix} as inputs
        \item Calculates the empirical moments for each observation
        \item Returns a $N \times L$ matrix of empirical moments
    \end{enumerate}
    You also have to give \texttt{gmm()} arguments for
    \begin{itemize}
        \item \texttt{x}: your data \emph{matrix}
        \item \texttt{t0}: parameter starting values
        \item Many other optional arguments
        \begin{itemize}
            \item \texttt{gmm()} can be very finicky with the optimization arguments
        \end{itemize}
    \end{itemize}
\end{frame}

\begin{frame}\frametitle{Calculating Empirical Moments in a Binary Logit Model}
    Steps to calculate this set of empirical moments for a given set of parameters in a binary logit model
    \begin{enumerate}
        \item Calculate the representative utility of the air conditioner for each decision maker.
        \item Calculate the choice probability of the air conditioner for each decision maker.
        \item Calculate the econometric residual, or the difference between the outcome and the probability, for each decision maker.
        \item Multiply this residual vector ($N \times 1$) by each column of the explanatory data matrix ($N \times K$).
        \item Return the resulting $N \times K$ matrix of empirical moments.
    \end{enumerate}
\end{frame}

\begin{frame}\frametitle{Binary Logit Model and Moments}
    The representative utility of the air conditioner is
    $$V_n = \beta_0 + \beta_1 SC_n + \beta_2 OC_n$$ \\
    \vspace{2ex}
    The choice probability is
    $$P_n = \frac{1}{1 + e^{-(\beta_0 + \beta_1 SC_n + \beta_2 OC_n)}}$$ \\
    \vspace{2ex}
    The empirical moments for each decision maker are
    $$[y_n - P_n(x_n, \hat{\beta})] x_n$$ \\
    \vspace{2ex}
    This model is just-identified, so we will actually use the MM estimator
    \begin{itemize}
        \item Three parameters and three moment conditions
    \end{itemize}
\end{frame}

\begin{frame}[fragile]\frametitle{Function to Calculate Moments}
    <<R CODE HERE>>
\end{frame}

\begin{frame}[fragile]\frametitle{Estimate MM Parameters}
    <<R CODE HERE>>
\end{frame}

\begin{frame}[fragile]\frametitle{Estimation Results}
    <<R CODE HERE>>
\end{frame}

\begin{frame}\frametitle{Binary Logit Model with Instruments}
    We may be concerned that the annual operating cost is endogenous
    $$E[OC_n \varepsilon_n] \neq 0$$ \\
    \begin{itemize}
        \item Why might this be true?
    \end{itemize}
    \vspace{3ex}
    We can re-write our moments with instruments instead of the explanatory variables
    \begin{align*}
        &E[[y_n - P_n(x_n, \beta_0)] z_n] = 0 \\
        \Rightarrow \quad &\frac{1}{N} \sum_{n = 1}^N [y_n - P_n(x_n, \hat{\beta})] z_n = 0
    \end{align*}
\end{frame}

\begin{frame}\frametitle{Instruments to include in model}
    Exogenous variables: constant term and system cost
    \begin{itemize}
        \item These variables are their own instruments
    \end{itemize}
    \vspace{2ex}
    Endogenous variable: operating cost
    \begin{itemize}
        \item Instrument with electricity price and square footage
    \end{itemize}
    \vspace{2ex}
    But now we have four moment conditions (one for each instrument) but only three parameters
    \begin{itemize}
        \item This model is now overidentified
        \item We will use optimal (two-step) GMM to estimate it
    \end{itemize}
\end{frame}

\begin{frame}[fragile]\frametitle{Function to Calculate Moments with Instruments}
    <<R CODE HERE>>
\end{frame}

\begin{frame}[fragile]\frametitle{Estimate GMM Parameters}
    <<R CODE HERE>>
\end{frame}

\begin{frame}[fragile]\frametitle{Estimation Results}
    <<R CODE HERE>>
\end{frame}

\begin{frame}[fragile]\frametitle{Test of Overidentifying Restrictions}
    <<R CODE HERE>>
\end{frame}

\begin{frame}\frametitle{Announcements}
    Reading for next time
    \begin{itemize}
        \item Train textbook, Chapter 4
    \end{itemize}
    \vspace{3ex}
    Upcoming
    \begin{itemize}
        \item Problem Set 3 will be posted soon, due October 31
    \end{itemize}
\end{frame}

\end{document}