\documentclass{beamer}
\usetheme{Boadilla}

\makeatother
\setbeamertemplate{footline}
{
    \leavevmode%
    \hbox{%
    \begin{beamercolorbox}[wd=.4\paperwidth,ht=2.25ex,dp=1ex,center]{author in head/foot}%
        \usebeamerfont{author in head/foot}\insertshortauthor
    \end{beamercolorbox}%
    \begin{beamercolorbox}[wd=.55\paperwidth,ht=2.25ex,dp=1ex,center]{title in head/foot}%
        \usebeamerfont{title in head/foot}\insertshorttitle
    \end{beamercolorbox}%
    \begin{beamercolorbox}[wd=.05\paperwidth,ht=2.25ex,dp=1ex,center]{date in head/foot}%
        \insertframenumber{}
    \end{beamercolorbox}}%
    \vskip0pt%
}
\makeatletter
\setbeamertemplate{navigation symbols}{}

\usepackage[T1]{fontenc}
\usepackage{lmodern}
\usepackage{amssymb,amsmath}
\renewcommand{\familydefault}{\sfdefault}

\usepackage{mathtools}
\usepackage{graphicx}
\usepackage{threeparttable}
\usepackage{booktabs}
\usepackage{siunitx}
\sisetup{parse-numbers=false}

%\setlength{\OuterFrameSep}{-2pt}
%\makeatletter
%\preto{\@verbatim}{\topsep=-10pt \partopsep=-10pt }
%\makeatother

\title[Lecture 8:\ Logit Estimation]{Lecture 8:\ Logit Estimation}
\author[ResEcon 703:\ Advanced Econometrics]{ResEcon 703:\ Topics in Advanced Econometrics}
\date{Matt Woerman\\University of Massachusetts Amherst}

\begin{document}

{\setbeamertemplate{footline}{} 
\begin{frame}[noframenumbering]
    \titlepage
\end{frame}
}

\begin{frame}\frametitle{Agenda}
    Last time
    \begin{itemize}
        \item Numerical Optimization
    \end{itemize}
    \vspace{2ex}
    Today
    \begin{itemize}
        \item Logit Estimation
        \item Logit Estimation Example in R
    \end{itemize}
    \vspace{2ex}
    Upcoming
    \begin{itemize}
        \item Reading for next time
        \begin{itemize}
            \item Greene textbook, Chapters 7.2.2--7.2.8 (Chapters 7.2.2--7.2.6 in the Seventh Edition)
        \end{itemize}
        \item Problem sets
        \begin{itemize}
            \item Problem Set 1 solutions are posted, will be returned soon
            \item Problem Set 2 is posted, due October 17
        \end{itemize}
    \end{itemize}
\end{frame}

\begin{frame}\frametitle{Course So Far}
    So far we have covered
    \begin{itemize}
    	\item Discrete choice framework
    	\item Random utility model
    	\item Logit model
    	\item Maximum likelihood estimation
    	\item Numerical optimization
    \end{itemize}
    \vspace{3ex}
    We can finally put all these pieces together and estimate a model ourselves!
\end{frame}

\begin{frame}\frametitle{}
    \vfill
    \centering
    \begin{beamercolorbox}[center]{title}
        \Large Logit Estimation
    \end{beamercolorbox}
    \vfill
\end{frame}

\begin{frame}\frametitle{Logit Likelihood Function}
    Under the assumptions of the logit model, the probability that decision maker $n$ chooses alternative $i$
    $$P_{ni} = \frac{e^{V_{ni}}}{\sum_{j = 1}^J e^{V_{nj}}}$$
    We can express the probability of decision maker $n$ choosing the alternative that was chosen, where $y_{ni} = 1$ if and only if $n$ chose $i$ 
    $$\prod_{i = 1}^J (P_{ni})^{y_{ni}}$$
    Then the likelihood function for the entire sample is
    $$L(\beta) = \prod_{n = 1}^N \prod_{i = 1}^J (P_{ni})^{y_{ni}}$$
\end{frame}

\begin{frame}\frametitle{Logit Log-Likelihood Function}
    Then the logit log-likelihood function is
    $$LL(\beta) = \sum_{n = 1}^N \sum_{i = 1}^J y_{ni} \ln P_{ni}$$
    The MLE is the set of parameters, $\hat{\beta}$, that maximize this function
\end{frame}

\begin{frame}\frametitle{Logit Moment Condition}
    The first-order condition for maximization is
    $$\frac{\partial LL(\beta)}{\partial \beta} = 0$$
    If we assume representative utility is linear, $V_{ni} = \beta' x_{ni}$, (and after some substitution and simplification) this first-order condition is equivalent to
    $$\sum_{n = 1}^N \sum_{i = 1}^J (y_{ni} - P_{ni}) x_{ni} = 0$$
    This is the sample covariance of the ``residuals,'' $y_{ni} - P_{ni}$, and the data, $x_{ni}$
    \begin{itemize}
    	\item This is a moment condition that we could use for GMM
    	\begin{itemize}
    		\item More on that in a couple weeks...
    	\end{itemize}
    \end{itemize}
\end{frame}

\begin{frame}\frametitle{Non-Random Sampling of Decision Makers}
    The previous slides all require that sampling is random (or at least exogenous to the choice)
    \begin{itemize}
    	\item If your sample is endogenous to the choice, then coefficients may be inconsistent
    	\item Example: What could go wrong if you take a survey of commute choices, but you take the survey at the bus stop?
    \end{itemize}
    \vspace{3ex}
    You can recover consistent estimates from a non-random sample, but that requires a more complex estimation procedure
\end{frame}

\begin{frame}\frametitle{}
    \vfill
    \centering
    \begin{beamercolorbox}[center]{title}
        \Large Logit Estimation Example in R
    \end{beamercolorbox}
    \vfill
\end{frame}

\begin{frame}\frametitle{Multinomial Logit Estimation Example}
    We are again studying how consumers make choices about expensive and highly energy-consuming systems in their homes. We have data on 900 households in California and the type of heating system in their home. Each household has the following choice set, and we observe the following data \\
    \vspace{3ex}
    \begin{columns}
    	\begin{column}{0.5\textwidth}
		    Choice set
		    \begin{itemize}
		    	\item GC: gas central
		    	\item GR: gas room
		    	\item EC: electric central
		    	\item ER: electric room
		    	\item HP: heat pump
		    \end{itemize}
		    \vspace{8ex}
	    \end{column}
	    \begin{column}{0.5\textwidth}
		    Alternative-specific data
		    \begin{itemize}
		    	\item IC: installation cost
		    	\item OC: annual operating cost
		    \end{itemize}
		    \vspace{2ex}
		    Household demographic data
		    \begin{itemize}
		    	\item income: annual income
		    	\item agehed: age of household head
		    	\item rooms: number of rooms
                \item region: home location
		    \end{itemize}
		\end{column}
    \end{columns}
\end{frame}

\begin{frame}[fragile]\frametitle{Load Dataset}
    <<R CODE HERE>>
\end{frame}

\begin{frame}[fragile]\frametitle{Dataset}
    <<R CODE HERE>>
\end{frame}

\begin{frame}\frametitle{Three Models to Estimate}
    \vspace{-4ex}
    \begin{align*}
        \intertext{Base model}
        U_{nj} &= \alpha_j + \beta_1 IC_{nj} + \beta_2 OC_{nj} + \varepsilon_{nj} \\
        \vspace{2ex}
        \intertext{Heterogeneous coefficients model}
        U_{nj} &= \alpha_j + \frac{\beta_1}{I_n} IC_{nj} + \frac{\beta_2}{I_n} OC_{nj} + \varepsilon_{nj} \\
        \vspace{2ex}
        \intertext{Alternative-specific coefficients model}
        U_{nj} &= \alpha_j + \beta_1 IC_{nj} + \beta_2 OC_{nj} + \beta_{3j} R_n + \varepsilon_{nj}
    \end{align*}
\end{frame}

\begin{frame}[fragile]\frametitle{Optimization in R}
    <<R CODE HERE>>
    \vspace{2ex}
    \texttt{optim()} requires that you create a function, \texttt{fn}, that
    \begin{enumerate}
        \item Takes a set of parameters and data as inputs
        \item Calculates your objective function given those parameters
        \item Returns this value of the objective function
    \end{enumerate}
    \vspace{2ex}
    You also have to give \texttt{optim()} arguments for
    \begin{itemize}
        \item \texttt{par}: starting parameter values
        \item \texttt{\ldots}: dataset and other things needed by your function
        \item \texttt{method}: optimization algorithm
        \begin{itemize}
            \item I recommend \texttt{method = 'BFGS'} for our estimation
        \end{itemize}
    \end{itemize}
\end{frame}

\begin{frame}\frametitle{Calculating the Log-Likelihood}
    Steps to calculate the value of the log-likelihood function for a given set of parameters
    \begin{enumerate}
        \item Calculate the representative utility for each alternative and for each decision maker.
        \item Calculate the choice probability of the chosen alternative for each decision maker.
        \item Sum the log of these choice probabilities to get the log-likelihood.
        \item Return the negative of the log-likelihood.
    \end{enumerate}
\end{frame}

\begin{frame}\frametitle{Base Model}
    $$U_{nj} = \alpha_j + \beta_1 IC_{nj} + \beta_2 OC_{nj} + \varepsilon_{nj}$$
\end{frame}

\begin{frame}[fragile]\frametitle{Function to Calculate Log-Likelihood}
    \vspace{1ex}
    <<R CODE HERE>>
\end{frame}

\begin{frame}[fragile]\frametitle{Estimate Logit Model}
    <<R CODE HERE>>
\end{frame}

\begin{frame}[fragile]\frametitle{Estimation Results}
    <<R CODE HERE>>
\end{frame}

\begin{frame}[fragile]\frametitle{Maximum Likelihood Estimator and Hessian}
    <<R CODE HERE>>
\end{frame}

\begin{frame}[fragile]\frametitle{Variance-Covariance Matrix}
    <<R CODE HERE>>
\end{frame}

\begin{frame}[fragile]\frametitle{Standard Errors}
    <<R CODE HERE>>
\end{frame}

\begin{frame}\frametitle{Heterogeneous Coefficients Model}
    $$U_{nj} = \alpha_j + \frac{\beta_1}{I_n} IC_{nj} + \frac{\beta_2}{I_n} OC_{nj} + \varepsilon_{nj}$$
\end{frame}

\begin{frame}[fragile]\frametitle{Heterogeneous Coefficients Estimation}
    \vspace{1ex}
    <<R CODE HERE>>
\end{frame}

\begin{frame}[fragile]\frametitle{Heterogeneous Coefficients Optimization}
    <<R CODE HERE>>
\end{frame}

\begin{frame}[fragile]\frametitle{Heterogeneous Coefficients MLE Results}
    <<R CODE HERE>>
\end{frame}

\begin{frame}\frametitle{Alternative-Specific Coefficients Model}
    $$U_{nj} = \alpha_j + \beta_1 IC_{nj} + \beta_2 OC_{nj} + \beta_{3j} R_n + \varepsilon_{nj}$$
\end{frame}

\begin{frame}[fragile]\frametitle{Alternative-Specific Coefficients Estimation}
    <<R CODE HERE>>
\end{frame}

\begin{frame}[fragile]\frametitle{Alternative-Specific Coefficients Estimation Data}
    <<R CODE HERE>>
\end{frame}

\begin{frame}[fragile]\frametitle{Alternative-Specific Coefficients Optimization}
    <<R CODE HERE>>
\end{frame}

\begin{frame}[fragile]\frametitle{Alternative-Specific Coefficients MLE Results}
    <<R CODE HERE>>
\end{frame}

\begin{frame}[fragile]\frametitle{Likelihood Ratio Test for Joint Significance}
    <<R CODE HERE>>
\end{frame}

\begin{frame}\frametitle{}
    \vfill
    \centering
    \begin{beamercolorbox}[center]{title}
        \Large Bayer et al. (2009)
    \end{beamercolorbox}
    \vfill
\end{frame}

\begin{frame}\frametitle{Bayer et al. (2009)}
    Research question
    \begin{itemize}
        \item How much do people value air quality in the place where they choose to live?
    \end{itemize}
    \vspace{2ex}
    Empirical methods
    \begin{itemize}
      \item Model the location choice of households as a multinomial logit model to recover composite city-level valuations
      \item Regress the composite city-level value on city attributes, including PM10 concentration, to estimate the value of air quality
    \end{itemize}
    \vspace{2ex}
    Results
    \begin{itemize}
      \item Willingness to pay for air quality is \$149--\$185 per $\mu \text{g} / \text{m}^3$ of PM10, or an elasticity of 0.34--0.42
    \end{itemize}
\end{frame}

\begin{frame}\frametitle{Residential Location Choice}
    The hedonic pricing model is the standard model used to estimate the value of air quality
    \begin{itemize}
      \item The hedonic model is based on the idea that a home is a bundle of attributes, each of which contributes to the price of the home, so we can estimate attribute values from home prices
      \item An underlying assumption of this model is that moving is costless
    \end{itemize}
    \vspace{2ex}
    An alternative approach developed by Bayer et al.\ uses multinomial logit
    \begin{itemize}
      \item The choice of where to live is clearly a discrete choice
      \item But air pollution is endogenous, and it is challenging to instrument within a multinomial model
      \item Bayer et al.\ develop this approach to first estimate a composite city-specific valuation, and then regress this value on attributes, which does allow for IV
    \end{itemize}
\end{frame}

\begin{frame}\frametitle{General Comments and Questions}
    Clever combination of structural estimation (first stage) and a linear IV model (second stage)
    \begin{itemize}
      \item This is becoming more common in the absence of exogenous variation: recover structural parameters and then deal with endogeneity
      \item Or you could use GMM...more on this later
    \end{itemize}
    \vspace{2ex}
    Some assumptions and implications of this model
    \begin{itemize}
      \item Income counterfactuals can be estimated from others with similar characteristics
      \item All households spend the same fraction of income on housing (20\%)
      \item Standard logit assumption of i.i.d.\ errors
    \end{itemize}
    \vspace{2ex}
    Do we believe
    \begin{itemize}
      \item These assumptions?
      \item The IV strategy?
    \end{itemize}
\end{frame}

\begin{frame}\frametitle{Announcements}
    Reading for next time
    \begin{itemize}
        \item Greene textbook, Chapters 7.2.2--7.2.8 (Chapters 7.2.2--7.2.6 in the Seventh Edition)
    \end{itemize}
    \vspace{3ex}
    Office hours
    \begin{itemize}
        \item Reminder: Tuesdays     at 2:00--3:00 in 218 Stockbridge
    \end{itemize}
    \vspace{3ex}
    Upcoming
    \begin{itemize}
        \item Problem Set 2 is posted, due October 17
    \end{itemize}
\end{frame}

\end{document}