\documentclass{beamer}
\usetheme{Boadilla}

\makeatother
\setbeamertemplate{footline}
{
    \leavevmode%
    \hbox{%
    \begin{beamercolorbox}[wd=.4\paperwidth,ht=2.25ex,dp=1ex,center]{author in head/foot}%
        \usebeamerfont{author in head/foot}\insertshortauthor
    \end{beamercolorbox}%
    \begin{beamercolorbox}[wd=.55\paperwidth,ht=2.25ex,dp=1ex,center]{title in head/foot}%
        \usebeamerfont{title in head/foot}\insertshorttitle
    \end{beamercolorbox}%
    \begin{beamercolorbox}[wd=.05\paperwidth,ht=2.25ex,dp=1ex,center]{date in head/foot}%
        \insertframenumber{}
    \end{beamercolorbox}}%
    \vskip0pt%
}
\makeatletter
\setbeamertemplate{navigation symbols}{}

\usepackage[T1]{fontenc}
\usepackage{lmodern}
\usepackage{amssymb,amsmath}
\renewcommand{\familydefault}{\sfdefault}

\DeclareMathOperator*{\argmax}{argmax}

\usepackage{mathtools}
\usepackage{graphicx}
\usepackage{threeparttable}
\usepackage{booktabs}
\usepackage{siunitx}
\sisetup{parse-numbers=false}

%\setlength{\OuterFrameSep}{-2pt}
%\makeatletter
%\preto{\@verbatim}{\topsep=-10pt \partopsep=-10pt }
%\makeatother

\title[Lecture 17:\ Simulation-Based Estimation I]{Lecture 17:\ Simulation-Based Estimation I}
\author[ResEcon 703:\ Advanced Econometrics]{ResEcon 703:\ Topics in Advanced Econometrics}
\date{Matt Woerman\\University of Massachusetts Amherst}

\begin{document}

{\setbeamertemplate{footline}{} 
\begin{frame}[noframenumbering]
    \titlepage
\end{frame}
}

\begin{frame}\frametitle{Agenda}
    Last time
    \begin{itemize}
        \item Mixed Logit Model Example in R
    \end{itemize}
    \vspace{2ex}
    Today
    \begin{itemize}
    	\item Simulated Choice Probabilities
        \item Simulation-Based Estimators
        \item Simulation Technicalities
        \item Simulation Bias and Noise
    \end{itemize}
    \vspace{2ex}
    Upcoming
    \begin{itemize}
        \item Reading for next time
        \begin{itemize}
            \item Optional: Handel (2013)
        \end{itemize}
        \item Problem sets
        \begin{itemize}
            \item Problem Set 4 is posted, due November 21
        \end{itemize}
    \end{itemize}
\end{frame}

\begin{frame}\frametitle{Mixed Logit Model}
    Mixed logit choice probability
    $$P_{ni} = \int \frac{e^{\beta' x_{ni}}}{\sum_{j = 1}^J e^{\beta' x_{nj}}} f(\beta \mid \theta) d \beta$$
    \begin{itemize}
        \item This integral does not have a closed-form expression
        \item We cannot estimate a mixed logit model using MLE, as we did for logit and nested logit
        \item Instead, we approximate choice probabilities through simulation and estimate the maximum simulated likelihood estimator (MSLE)
    \end{itemize}
    \vspace{2ex}
    How do we simulate mixed logit choice probabilities? \\
    \vspace{2ex}
    What is maximum simulated likelihood estimation?
\end{frame}

\begin{frame}\frametitle{}
    \vfill
    \centering
    \begin{beamercolorbox}[center]{title}
        \Large Simulated Choice Probabilities
    \end{beamercolorbox}
    \vfill
\end{frame}

\begin{frame}\frametitle{Simulated Choice Probabilities}
    The mixed logit choice probability is
    \begin{align*}
        P_{ni} &= \int L_{ni}(\beta) f(\beta \mid \theta) d \beta \\
        \intertext{where $L_{ni}(\beta)$ is the logit choice probability at a given set of coefficients, $\beta$,}
        L_{ni}(\beta) &= \frac{e^{\beta' x_{ni}}}{\sum_{j = 1}^J e^{\beta' x_{nj}}}
    \end{align*}
    To simulate this choice probability for a given set of parameters, $\theta$,
    \begin{enumerate}
        \item Draw a set of coefficients, $\beta^r$, from the density $f(\beta \mid \theta)$
        \item Calculate the conditional probability, $L_{ni}(\beta^r)$, for each alternative
        \item Repeat steps 1 and 2 for a total of R draws from $f(\beta \mid \theta)$
        \item Average over these R draws to get $\check{P}_{ni}$ for each alternative
        $$\check{P}_{ni} = \frac{1}{R} \sum_{r = 1}^R L_{ni}(\beta^r)$$
    \end{enumerate}
\end{frame}

\begin{frame}\frametitle{}
    \vfill
    \centering
    \begin{beamercolorbox}[center]{title}
        \Large Simulated-Based Estimators
    \end{beamercolorbox}
    \vfill
\end{frame}

\begin{frame}\frametitle{Simulation-Based Estimation}
    Simulation-based estimators are roughly equivalent to their traditional analogs
    \begin{itemize}
        \item We replace terms that are difficult or impossible to calculate with their simulated counterparts
        \begin{itemize}
            \item Example: Mixed logit choice probabilities include an integral and do not have a closed-form expression, so we replace them with simulated choice probabilities
        \end{itemize}
        \item Simulation can potentially introduce bias or noise into the estimation, which we need to consider when using simulation-based estimators
    \end{itemize}
\end{frame}

\begin{frame}\frametitle{Maximum Simulated Likelihood Estimation}
    Maximum simulated likelihood estimation (MSLE), or simulated maximum likelihood estimation (SMLE), is the simulation analog of maximum likelihood estimation (MLE) \\
    \vspace{2ex}
    MSLE is the estimator that maximizes the simulated log-likelihood
    \begin{align*}
        \hat{\theta} &= \argmax_\theta \ln \check{L}(\theta \mid y) \\
        \intertext{where $\ln \check{L}(\theta \mid y)$ is the log of the simulated likelihood}
        \ln \check{L}(\theta \mid y) &= \sum_{n = 1}^N \ln \check{f}(y_n \mid \theta)
    \end{align*}
    and $\check{f}(y_n \mid \theta)$ is a simulated density function
\end{frame}

\begin{frame}\frametitle{MSLE with Discrete Choice}
    For discrete choice applications, the log of simulated likelihood is a function of simulated choice probabilities
    \begin{align*}
        \ln \check{L}(\theta \mid y) &= \sum_{n = 1}^N \sum_{j = 1}^J y_{nj} \ln \check{P}_{nj}(\theta) \\
        \intertext{so the MSLE estimator is}
        \hat{\theta} &= \argmax_\theta \sum_{n = 1}^N \sum_{j = 1}^J y_{nj} \ln \check{P}_{nj}(\theta) \\
        \intertext{which yields the first-order condition}
        0 &= \sum_{n = 1}^N \frac{\partial \ln \check{P}_n(\hat{\theta})}{\partial \theta}
    \end{align*}
    where $ \check{P}_n(\theta)$ is the simulated choice probability of the chosen alternative
\end{frame}

\begin{frame}\frametitle{Method of Simulated Moments}
    Method of simulated moments (MSM), or simulated method of moments (SMM), is the simulation analog of generalized method of moments (GMM) \\
    \vspace{2ex}
    MSM is the estimator that ``solves'' simulated moments
    \begin{align*}
        \frac{1}{N} \sum_{n = 1}^N \check{m}(w_n, \hat{\theta}) &= 0
        \intertext{where $\check{m}(w_n, \hat{\theta})$ are the simulated empirical analogs of population moments that satisfy}
        E[m(w_n, \theta^*)] &= 0
    \end{align*}
    With more moments than parameters, we cannot actually solve the first equation, so we minimize a weighted function of the moments
\end{frame}

\begin{frame}\frametitle{MSM with Discrete Choice}
    For discrete choice applications, the population moments result from the model residuals being uncorrelated with a set of exogenous instruments
    \begin{align*}
        E \left[ \sum_{j = 1}^J [y_{nj} - P_{nj}(\theta^*)] z_{nj} \right] &= 0 \\
        \intertext{so the MSM estimator ``solves'' the simulated empirical analogs}
        \frac{1}{N} \sum_{n = 1}^N \sum_{j = 1}^J [y_{nj} - \check{P}_{nj}(\hat{\theta})] z_{nj} &= 0
    \end{align*}
\end{frame}

\begin{frame}\frametitle{}
    \vfill
    \centering
    \begin{beamercolorbox}[center]{title}
        \Large Simulation Technicalities
    \end{beamercolorbox}
    \vfill
\end{frame}

\begin{frame}\frametitle{Some Technicalities of Simulating Choice Probabilities}
	$$\check{P}_{ni} = \frac{1}{R} \sum_{r = 1}^R L_{ni}(\beta^r)$$ \\
	\vspace{2ex}
    For a given set of coefficient density parameters, $\theta$, perform simulate choice probabilities for every alternative for each decision maker
    \begin{itemize}
        \item For a given decision maker, use the same set of $\beta^r$ draws to simulate the choice probability for every alternative
        \begin{itemize}
            \item That is, for each draw of $\beta^r$, calculate the full set of $J$ choice probabilities (one for each alternative), and then average each choice probability over the same $R$ draws of $\beta^r$
        \end{itemize}
        \item Use a different set of $\beta^r$ draws for each decision maker in order to maintain independence over decision makers
        \begin{itemize}
            \item That is, we need $N$ different sets of $R$ draws of $\beta^r$
        \end{itemize}
    \end{itemize}
\end{frame}

\begin{frame}\frametitle{Simulated Choice Probabilities and Numerical Optimization}
    We use these simulated choice probabilities within a numerical optimization algorithm to find the set of parameters, $\hat{\theta}$, that are our estimators
    \begin{itemize}
        \item We want to use the same set of ``draws'' for a given decision maker throughout the numerical optimization algorithm
        \item If we use different ``draws'' for each iteration of the algorithm, we introduce additional noise that impedes convergence
    \end{itemize}
    \vspace{3ex}
     In order to avoid this noise
    \begin{enumerate}
        \item Draw many ($K \times N \times R$) random variables from a standard normal distribution before starting the optimization algorithm
        \item Transform this same set of standard normal random variables in each iteration of the optimization algorithm to represent $f(\beta \mid \theta)$ for the set of parameters, $\theta$, of each iteration
    \end{enumerate}
\end{frame}

\begin{frame}\frametitle{Transforming a Standard Normal Random Variable}
    We can transform a standard normal random variable (or a vector of standard normals) into many other distributions
    \begin{enumerate}
        \item Draw $K$ standard normal random variables, $\eta \sim N(0, 1)$, where $K$ is the number of random parameters
        \item Transform these standard normals into the desired distributions
        \begin{itemize}
            \item Normal: $\varepsilon = b + s \eta$ gives $\varepsilon \sim N(b, s^2)$
            \item Log-normal: $\varepsilon = e^{b + s \eta}$ gives $\ln \varepsilon \sim N(b, s^2)$
            \item Multivariate normal: $\varepsilon = b + L \eta$ gives $\varepsilon \sim N(b, \Omega)$ where $\varepsilon$, $b$, and $\eta$ are each a vector of length equal to the number of multivariate normal random variables, $\Omega$ is the variance-covariance matrix of these variables, and $L$ is the Choleski factor of $\Omega$
            \item Comparable transformations exist for most distributions
        \end{itemize}
    \end{enumerate}
    \vspace{2ex}
    See Chapter 9 in the Train textbook for more on drawing and transforming random variables
\end{frame}

\begin{frame}\frametitle{Steps for Simulation-Based Estimation}
    \begin{enumerate}
        \item Draw $K \times N \times R$ standard normal random variables
        \begin{itemize}
            \item $K$ random coefficients for each of
            \item $N$ different decision makers for each of
            \item $R$ different simulation draws
        \end{itemize}
        \item Find the set of parameters that maximizes or minimizes the objective function of a simulation-based estimator
        \begin{enumerate}
            \item Start with some set of parameters, $\theta^0$
            \item Simulate choice probabilities for this set of parameters, $\check{P}_{ni}(\theta^t)$
            \begin{enumerate}
                \item Transform each set of $K$ standard normals using $\theta^t$ to get a set of $\beta_n^r$
                \item Calculate the choice probabilities for each individual and draw, $L_{ni}(\beta_n^r)$
                \item Average over all R simulation draws to get $\check{P}_{ni}(\theta^t)$
            \end{enumerate}
            \item Use these simulated choice probabilities to calculate simulated log-likelihood, simulated moments, etc.
            \item Step to a better set of parameters, $\theta^{t + 1}$
            \item Repeat steps 2 and 3 until the algorithm converges to a set of parameters that is your simulation-based estimator
        \end{enumerate}
    \end{enumerate}
\end{frame}

\begin{frame}\frametitle{}
    \vfill
    \centering
    \begin{beamercolorbox}[center]{title}
        \Large Simulated Bias and Noise
    \end{beamercolorbox}
    \vfill
\end{frame}

\begin{frame}\frametitle{Traditional Estimators}
    Traditional estimators are the set of parameters that solve a specific function
    \begin{align*}
        g(\hat{\theta}) &= \frac{1}{N} \sum_{n = 1}^N g_n(\hat{\theta}) = 0 \\
        \intertext{ Maximum likelihood estimation of a discrete choice model}
        g_n(\theta) &= \frac{\partial \ln P_n(\theta)}{\partial \theta} \\
        \intertext{Generalized method of moments estimation of a discrete choice model}
        g_n(\theta) &= \sum_{j = 1}^J [y_{nj} - P_{nj}(\theta)] z_{nj}
    \end{align*}
    When the some assumptions are met, these estimators yield consistent estimates of the true set of parameters, $\theta^*$
\end{frame}

\begin{frame}\frametitle{Simulation-Based Estimators}
     Simulation-based estimators are the set of parameters that solve a specific function
     $$\check{g}(\hat{\theta}) = \frac{1}{N} \sum_{n = 1}^N \check{g}_n(\hat{\theta}) = 0$$
     where $\check{g}_n(\theta)$ is the simulation-based analog of $g_n(\theta)$ \\
     \vspace{2ex}
     We can compare this function for the simulation-based and traditional estimators by writing
     \begin{align*}
        \check{g}(\theta) &= \check{g}(\theta) + \{ g(\theta) - g(\theta) \} + \{ E_r[\check{g}(\theta)] - E_r[\check{g}(\theta)] \} \\
        &= g(\theta) + \{ E_r[\check{g}(\theta)] - g(\theta) \} + \{ \check{g}(\theta) - E_r[\check{g}(\theta)] \}
     \end{align*} \\
     \vspace{2ex}
     A simulation-based estimator can differ from a traditional estimator for two reasons
     \begin{itemize}
         \item Simulation bias: $E_r[\check{g}(\theta)] - g(\theta)$
         \item Simulation noise: $\check{g}(\theta) - E_r[\check{g}(\theta)]$
     \end{itemize}
\end{frame}

\begin{frame}\frametitle{Simulation Bias and Noise}
    How do we reduce the simulation bias and noise in a simulation-based estimator?
    \begin{itemize}
        \item Increase the sample size, $N$
        \item Increase the number of simulation draws, $R$
    \end{itemize}
    \vspace{2ex}
    With sufficient sample size and simulation draws, a simulation-based estimator is:
    \begin{itemize}
        \item Consistent
        \item Asymptotically normal
        \item Sometimes equivalent (or converging) to its traditional analog
    \end{itemize}
    \vspace{2ex}
    The specifics depend on the estimator in question
    \begin{itemize}
        \item See Chapter 10 in the Train textbook for details
    \end{itemize}
\end{frame}

\begin{frame}\frametitle{Announcements}
    Reading for next time
    \begin{itemize}
        \item Optional: Handel (2013)
    \end{itemize}
    \vspace{3ex}
    Office hours
    \begin{itemize}
    	\item Reminder: 2:00--3:00 on Tuesdays in 218 Stockbridge
    \end{itemize}
    \vspace{3ex}
    Upcoming
    \begin{itemize}
        \item Problem Set 4 is posted, due November 21
    \end{itemize}
\end{frame}

\end{document}